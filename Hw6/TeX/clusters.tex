\documentclass{standalone}
\usepackage{tikz}
\usetikzlibrary{calc}
\usetikzlibrary{shapes}
\usepackage{amsmath}

\usepackage{fontspec}
\setsansfont{CMU Sans Serif}%{Arial}
\setmainfont{CMU Serif}%{Times New Roman}
\setmonofont{CMU Typewriter Text}%{Consolas}
\defaultfontfeatures{Ligatures={TeX}}
\usepackage[math-style=TeX]{unicode-math}
\usepackage[english, russian, ukrainian]{babel}

\begin{document}
\begin{tikzpicture}
  % Generate random points around cluster centers
  \foreach \centerx/\centery in {2/2, 6/4, 4/6}
  {
    \foreach \i in {1,...,50}
    {
      \pgfmathsetmacro\x{rnd*2 + \centerx - 1}
      \pgfmathsetmacro\y{rnd*2 + \centery - 1}
      \filldraw[blue] (\x,\y) circle (1pt);

      % Calculate distance to centroid
      \pgfmathsetmacro\dist{sqrt((\x-\centerx)^2 + (\y-\centery)^2)}
%      \node[font=\tiny] at (\x,\y) [above] {$\dist$};

      % Color code points based on cluster
      \ifdim \dist pt < 1.0 pt
        \fill[red] (\x,\y) circle (1pt);
      \else
        \ifdim \dist pt < 2.0 pt
          \fill[green] (\x,\y) circle (1pt);
        \else
          \fill[orange] (\x,\y) circle (1pt);
        \fi
      \fi
    }
  }

  % Define cluster centers
  \foreach \centerx/\centery in {2/2, 6/4, 4/6}
    \filldraw[red] (\centerx,\centery) circle (2pt);

  % Cluster labels
  \node[align=center] at (2, 0.8) {Cluster 1};
  \node[align=center] at (6, 2.8) {Cluster 2};
  \node[align=center] at (4, 4.8) {Cluster 3};

  % Axes
  \draw[->] (0,0) -- (8,0) node[below] {Feature $1$};
  \draw[->] (0,0) -- (0,7) node[above] {Feature $2$};
\end{tikzpicture}
\end{document}
