% !TeX program = lualatex
% !TeX encoding = utf8
% !TeX spellcheck = uk_UA

\documentclass[]{article}
\usepackage[fontsize = 12pt]{fontsize}
\usepackage{ifluatex}

\ifluatex
	\usepackage{fontspec}
	\setsansfont{CMU Sans Serif}%{Arial}
	\setmainfont{CMU Serif}%{Times New Roman}
	\setmonofont{CMU Typewriter Text}%{Consolas}
	\defaultfontfeatures{Ligatures={TeX}}
	\usepackage[math-style=TeX]{unicode-math}
\else
	\usepackage[utf8]{inputenc}
	\usepackage[T2A,T1]{fontenc}
	\usepackage{cmap}
\fi
\usepackage[english, russian, ukrainian]{babel}

\usepackage[%
	a4paper,%
	footskip=1cm,%
	headsep=0.3cm,%
	top=2cm, %поле сверху
	bottom=2cm, %поле снизу
	left=2cm, %поле ліворуч
	right=2cm, %поле праворуч
    ]{geometry}
\renewcommand{\baselinestretch}{1}

\clubpenalty =10000
\widowpenalty=10000

\setlength{\parskip}{0.5ex}%
\setlength{\parindent}{2.5em}%

\usepackage{titlesec}


\titleformat{\section}{\normalfont\bfseries}{\thesection.}{1em}{}
\titleformat{\subsection}{\small\bfseries}{\thesubsection.}{1em}{}

\usepackage{tikz, pgfplots}

\usepackage[%colorlinks=true,
	%urlcolor = blue, %Colour for external hyperlinks
	%linkcolor  = malina, %Colour of internal links
	%citecolor  = green, %Colour of citations
	bookmarks = true,
	bookmarksnumbered=true,
	unicode,
	linktoc = all,
	hypertexnames=false,
	pdftoolbar=false,
	pdfpagelayout=TwoPageRight,
	pdfauthor={Ponomarenko S.M. aka sergiokapone},
	pdfdisplaydoctitle=true,
	pdfencoding=auto
	]%
	{hyperref}
		\makeatletter
	\AtBeginDocument{
	\hypersetup{
		pdfinfo={
		Title={\@title},
		}
	}
	}
	\makeatother
\usepackage{amsmath}

\usepackage[breakable, most, minted]{tcolorbox}

\usepackage{minted}

\newcounter{pythoncode}
\newtcblisting{pythoncode}{
    listing engine=minted,
    listing only,
    sharp corners,
    breakable,
    size=fbox,
    boxrule=1pt,
    pad at break*=1mm,
    colback=gray!5,
    colframe=gray!50,
    enhanced,
    minted language=python,
    minted options={
        autogobble,
        fontsize=\small,
        breaklines,
        style=emacs
    },
    overlay={
        \node[anchor=north east, font=\ttfamily\scriptsize, color=gray] at (frame.north west) {In [\thepythoncode]:\relax};
    },
    code={\stepcounter{pythoncode}}
}


\newtcblisting{out}{
    listing engine=minted,
    listing only,
    sharp corners,
    breakable,
    size=fbox,
    boxrule=1pt,
    pad at break*=1mm,
    colback=white,
    colframe=white,
    enhanced,
    minted language=python,
    minted options={
        autogobble,
        fontsize=\small,
        breaklines
    },
    overlay={
        \node[anchor=north east, font=\ttfamily\scriptsize, color=gray] at (frame.north west) {Out[\thepythoncode]:\relax};
    },
}



\title{Знайомство з Numpy}
\begin{document}

\maketitle





\section*{Імпорт модулів}


\begin{pythoncode}
	import numpy as np
\end{pythoncode}





\section{Створіть одновимірний масив (вектор) з першими 10-ма
  натуральними числами та виведіть його значення.}

Натуральні числа --- це числа, що виникають природним чином при лічбі.
Це числа: \(1\), \(2\), \(3\), \(4\), \(\ldots\) Множину натуральних
чисел прийнято позначати \(\mathbb{N}\).


\begin{pythoncode}
	vector = np.arange(1, 11)
	print(vector)
\end{pythoncode}


\begin{out}
[ 1  2  3  4  5  6  7  8  9 10]
\end{out}

\section{Створіть двовимірний масив (матрицю) розміром 3x3,
  заповніть його нулями та виведіть його
  значення.}

\begin{pythoncode}
	matrix = np.zeros((3, 3))
	print(matrix)
\end{pythoncode}

\begin{out}
[[0. 0. 0.]
 [0. 0. 0.]
 [0. 0. 0.]]
\end{out}

\section{Створіть масив розміром 5x5, заповніть його випадковими цілими числами в діапазоні від 1 до 10 та виведіть його значення}

\begin{pythoncode}
    random_matrix = np.random.randint(1, 11, (5, 5))
    print(random_matrix)
\end{pythoncode}

\begin{out}
	[[ 7 10  4  6  7]
	 [ 1 10  1  1 10]
	 [10  9 10  6  3]
	 [10  7  9  6  6]
	 [ 9  2  4  4  4]]
\end{out}

\section{Створіть масив розміром 4x4, заповніть його випадковими дійсними числами в діапазоні від 0 до 1 та виведіть його значення.}

\begin{pythoncode}
    random_matrix = np.random.rand(4, 4)
    print(random_matrix)
\end{pythoncode}

\begin{out}
	[[0.34716231 0.38666569 0.60847232 0.44947206]
	 [0.28441679 0.24889235 0.69068513 0.44390396]
	 [0.45757561 0.30179135 0.30334876 0.14319334]
	 [0.19297175 0.18552924 0.3236188  0.56707376]]
\end{out}

\section{Створіть два одновимірних масиви розміром 5, заповніть  	випадковими цілими числами в діапазоні від 1 до 10 та виконайте на них 	поелементні операції додавання, віднімання та множення.}

\subsection{Створення масивів}

\begin{pythoncode}
    array1 = np.random.randint(1, 11, 5)
    array2 = np.random.randint(1, 11, 5)

    print("Масив 1:", array1)
    print("Масив 2:", array2)
\end{pythoncode}

\begin{out}
	Масив 1: [1 5 1 9 9]
	Масив 2: [8 2 4 6 7]
\end{out}

\subsection{Поелементні операції}

\begin{pythoncode}
addition_result = array1 + array2
print("Результат додавання:", addition_result)

# Виконаємо поелементну операцію віднімання
subtraction_result = array1 - array2
print("Результат віднімання:", subtraction_result)

# Виконаємо поелементну операцію множення
multiplication_result = array1 * array2
print("Результат множення:", multiplication_result)
\end{pythoncode}

\begin{out}
	Результат додавання: [ 9  7  5 15 16]
	Результат віднімання: [-7  3 -3  3  2]
	Результат множення: [ 8 10  4 54 63]
\end{out}

\section{Створіть два вектори розміром 7, заповніть довільними 	числами та знайдіть їх скалярний добуток}

Скалярний добуток (scalar or dot product) можна обчислити за наступною
формулою
\(\vec v \cdot \vec u = v^1 u_1 + v^2 u_2 + v^3 u_3 + \ldots + v^n u_n\)

\begin{pythoncode}
    vector1 = np.random.random(7)
    vector2 = np.random.random(7)

    scalar_product = np.dot(vector1, vector2)

    print("Скалярний добуток векторів:", scalar_product)
\end{pythoncode}

\begin{out}
	Скалярний добуток векторів: 1.8680632685208063
\end{out}

\section{Створіть дві матриці розміром 2x2 та 2x3, заповніть їх випадковими цілими числами в діапазоні від 1 до 10 та перемножте їх між собою.}

\begin{pythoncode}
    matrix1 = np.random.randint(1, 11, (2, 2))
    matrix2 = np.random.randint(1, 11, (2, 3))

    print("Матриця 1:")
    print(matrix1)
    print("Матриця 2:")
    print(matrix2)

    result = np.dot(matrix1, matrix2)

    print("Результат множення:")
    print(result)
\end{pythoncode}

\begin{out}
	Матриця 1:
	[[ 8  6]
	 [10  4]]
	Матриця 2:
	[[ 7 10  4]
	 [ 6 10  7]]
	Результат множення:
	[[ 92 140  74]
	 [ 94 140  68]]
\end{out}

\section{Створіть матрицю розміром 3x3, заповніть її випадковими цілими числами в діапазоні від 1 до 10 та знайдіть її обернену матрицю.}

Обчислення оберненої матриці здійснюється за формулою:

\[ \mathrm{A}^{-1} = \frac{1}{\mathrm{det}(\mathrm{A})} \cdot \mathrm{adj}(\mathrm{A}), \]

де $ \mathrm{det}(A) $ --- визначник матриці \(\mathrm A\),
\(\mathrm{adj}(\mathrm{A})\) --- доповнена матриця.

Обернена матриця існує, якщо $ \mathrm{det}(A) \neq 0$. Матриця, для
якої $ \mathrm{det}(A) = 0$ називається сингулярною, і для такої
матриці оберненої не існує. Тому в коді ми маємо це перевірити.

\begin{pythoncode}
    # Створимо матрицю розміром 3x3 з випадковими цілими числами від 1 до 10
    matrix = np.random.randint(1, 11, (3, 3))

    # Виведемо початкову матрицю
    print("Початкова матриця:")
    print(matrix)

    # Знайдемо обернену матрицю
    try:
        inverse_matrix = np.linalg.inv(matrix)
        print("Обернена матриця:")
        print(inverse_matrix)
    except np.linalg.LinAlgError:
        print("Матриця не має оберненої матриці.")
\end{pythoncode}

\begin{out}
	Початкова матриця:
	[[ 2  4  6]
	 [ 7  1  1]
	 [ 5  1 10]]
	Обернена матриця:
	[[-0.03913043  0.14782609  0.00869565]
	 [ 0.2826087   0.04347826 -0.17391304]
	 [-0.00869565 -0.07826087  0.11304348]]
\end{out}

\section{Створіть матрицю розміром 4x4, заповніть її випадковими дійсними числами в діапазоні від 0 до 1 та транспонуйте її}

Транспонована матриця \(\mathrm{A}^T\), виникає з матриці \(\mathrm{A}\)
в результаті заміни рядків на стовпчики $(\mathrm{A}^{T})_{ji} =
	\mathrm{A}_{ij} $:

\[A =
	\begin{bmatrix}
		1 & 2 & 3 \\
		4 & 5 & 6 \\
	\end{bmatrix}
\]

Тоді транспонована матриця $ A^{T} $ буде мати вигляд:

\[
	A^T =
	\begin{bmatrix}
		1 & 4 \\
		2 & 5 \\
		3 & 6 \\
	\end{bmatrix}
\]

\begin{pythoncode}
    matrix = np.random.rand(4, 4)

    # Виведемо початкову матрицю
    print("Початкова матриця:")
    print(matrix)

    # Транспонуємо матрицю
    transposed_matrix = np.transpose(matrix)

    # Виведемо транспоновану матрицю
    print("Транспонована матриця:")
    print(transposed_matrix)
\end{pythoncode}

\begin{out}
	Початкова матриця:
	[[0.08402274 0.07052606 0.94668645 0.47278038]
	 [0.65529227 0.45497744 0.38751573 0.46614788]
	 [0.57033985 0.4134212  0.46341259 0.69720946]
	 [0.15384283 0.5854396  0.90123728 0.15706181]]
	Транспонована матриця:
	[[0.08402274 0.65529227 0.57033985 0.15384283]
	 [0.07052606 0.45497744 0.4134212  0.5854396 ]
	 [0.94668645 0.38751573 0.46341259 0.90123728]
	 [0.47278038 0.46614788 0.69720946 0.15706181]]
\end{out}

\section{Створіть матрицю розміром 3x4 та вектор розміром 4, заповніть їх випадковими цілими числами в діапазоні від 1 до 10 та перемножте матрицю на вектор.}
\[
	\mathbf{u} = \mathrm{A} \cdot \mathbf{v} =
	\begin{bmatrix}
		a_{11} & a_{12} & a_{13} \\
		a_{21} & a_{22} & a_{23} \\
		a_{31} & a_{32} & a_{33} \\
	\end{bmatrix}
	\begin{bmatrix}
		v_1 \\
		v_2 \\
		v_3 \\
	\end{bmatrix}
	=
	\begin{bmatrix}
		a_{11} \cdot v_1 + a_{12} \cdot v_2 + a_{13} \cdot v_3 \\
		a_{21} \cdot v_1 + a_{22} \cdot v_2 + a_{23} \cdot v_3 \\
		a_{31} \cdot v_1 + a_{32} \cdot v_2 + a_{33} \cdot v_3 \\
	\end{bmatrix}
\]

\begin{pythoncode}
    # Створимо матрицю розміром 3x4 з випадковими цілими числами від 1 до 10
    matrix = np.random.randint(1, 11, (3, 4))

    # Створимо вектор розміром 4 з випадковими цілими числами від 1 до 10
    vector = np.random.randint(1, 11, 4)

    # Виведемо матрицю і вектор
    print("Матриця:")
    print(matrix)
    print("Вектор:")
    print(vector)

    # Перемножимо матрицю на вектор
    result = np.dot(matrix, vector)

    # Виведемо результат
    print("Результат перемноження:")
    print(result)
\end{pythoncode}

\begin{out}
	Матриця:
	[[5 3 6 2]
	 [7 2 7 7]
	 [5 9 5 7]]
	Вектор:
	[ 8  3 10  9]
	Результат перемноження:
	[127 195 180]
\end{out}

\section{Створіть матрицю розміром 2x3 та вектор розміром 3, заповніть їх випадковими дійсними числами в діапазоні від 0 до 1 та перемножте матрицю на вектор.}

\begin{pythoncode}
    # Створимо матрицю розміром 2x3 з випадковими дійсними числами від 0 до 1
    matrix = np.random.rand(2, 3)

    # Створимо вектор розміром 3 з випадковими дійсними числами від 0 до 1
    vector = np.random.rand(3)

    # Виведемо матрицю і вектор
    print("Матриця:")
    print(matrix)
    print("Вектор:")
    print(vector)

    # Перемножимо матрицю на вектор
    result = np.dot(matrix, vector)

    # Виведемо результат
    print("Результат перемноження:")
    print(result)
\end{pythoncode}

\begin{out}
    Матриця:
    [[0.40142629 0.24681107 0.41057536]
     [0.72516117 0.91335118 0.68146542]]
    Вектор:
    [0.3660769  0.20024095 0.43328539]
    Результат перемноження:
    [0.37427088 0.74362408]
\end{out}

\section{Створіть дві матриці розміром 2x2, заповніть їх 	випадковими цілими числами в діапазоні від 1 до 10 та виконайте їхнє 	поелементне 	множення.}

Поелементне множення двох матриць визначається так: результатом цієї
операції є нова матриця, у якій кожний елемент результуючої матриці
визначається, як добуток відповідних елементів вихідних матриць.

Елемент результуючої матриці \(\mathrm C\) обчислюється як добуток
відповідних елементів матриць \(\mathrm A\) і \(\mathrm B\):

\[
	\mathrm C_{ij}  = \mathrm A_{ij} \cdot \mathrm B_{ij}
\]

де \(\mathrm C_{ij}\) --- елемент результуючої матриці \(\mathrm C\) на
позиції \((i, j)\), \(\mathrm A_{ij}\) --- елемент матриці \(\mathrm A\)
на позиції \((i, j)\) і \(\mathrm B_{ij}\) - елемент матриці
\(\mathrm B\) на позиції \((i, j)\).

Поелементне множення відрізняється від стандартного множення матриць,
оскільки в останньому випадку обчислюється добуток рядків однієї матриці
на стовпці іншої матриці, що веде до іншої результуючої матриці.

\begin{pythoncode}
    # Створимо першу матрицю 2x2 з випадковими цілими числами в діапазоні від 1 до 10
    matrix1 = np.random.randint(1, 11, (2, 2))

    # Створимо другу матрицю 2x2 з випадковими цілими числами в діапазоні від 1 до 10
    matrix2 = np.random.randint(1, 11, (2, 2))

    # Виведемо обидві матриці
    print("Перша матриця:")
    print(matrix1)
    print("Друга матриця:")
    print(matrix2)

    # Виконаємо поелементне множення матриць
    result = np.multiply(matrix1, matrix2)

    # Виведемо результат
    print("Результат поелементного множення:")
    print(result)
\end{pythoncode}

\begin{out}
	Перша матриця:
	[[5 7]
	 [1 2]]
	Друга матриця:
	[[ 7 10]
	 [ 3  9]]
	Результат поелементного множення:
	[[35 70]
	 [ 3 18]]
\end{out}

\section{Створіть дві матриці розміром 2x2, заповніть їх випадковими цілими числами в діапазоні від 1 до 10 та знайдіть їх добуток.}

Добуток матриць визначаться як:

\[\mathrm C_{ij} = \sum_{r=1}^m \mathrm A_{ir} \mathrm B_{rj} \;\;\; \left(i=1, 2, \ldots l;\; j=1, 2, \ldots n \right)\]

\begin{pythoncode}
    # Створимо першу матрицю 2x2 з випадковими цілими числами в діапазоні від 1 до 10
    matrix1 = np.random.randint(1, 11, (2, 2))

    # Створимо другу матрицю 2x2 з випадковими цілими числами в діапазоні від 1 до 10
    matrix2 = np.random.randint(1, 11, (2, 2))

    # Виведемо обидві матриці
    print("Перша матриця:")
    print(matrix1)
    print("Друга матриця:")
    print(matrix2)

    # Знайдемо добуток матриць
    result = np.dot(matrix1, matrix2)

    # Виведемо результат
    print("Результат добутку матриць:")
    print(result)
\end{pythoncode}

\begin{out}
	Перша матриця:
	[[1 3]
	 [2 5]]
	Друга матриця:
	[[7 7]
	 [2 3]]
	Результат добутку матриць:
	[[13 16]
	 [24 29]]
\end{out}

\section{Створіть матрицю розміром 5\textbackslash times 5, заповніть її випадковими цілими числами в діапазоні від 1 до 100 та знайдіть суму елементів матриці.}

\begin{pythoncode}
    # Створимо матрицю 5x5 з випадковими цілими числами від 1 до 100
    matrix = np.random.randint(1, 101, (5, 5))

    # Виведемо матрицю
    print("Матриця:")
    print(matrix)

    # Знайдемо суму елементів матриці
    sum_of_elements = np.sum(matrix)

    # Виведемо суму
    print("Сума елементів матриці: ", sum_of_elements)
\end{pythoncode}

\begin{Verbatim}[commandchars=\\\{\}]
	Матриця:
	[[ 65  54  46 100  89]
	 [ 19  41  79  28  70]
	 [ 73  27  94  20  94]
	 [ 46  35  18  68  27]
	 [ 36  63  33  76   7]]
	Сума елементів матриці:  1308
\end{Verbatim}

\section{\texorpdfstring{Створіть дві матриці розміром
		\(4\times 4\), заповніть їх випадковими цілими числами в діапазоні від 1
		до 10 та знайдіть їхню
		різницю.}{Створіть дві матриці розміром 4\textbackslash times 4, заповніть їх випадковими цілими числами в діапазоні від 1 до 10 та знайдіть їхню різницю.}}

Віднімання матриць \(\mathrm A - \mathrm B\) - це операція обчислення
матриці \(\mathrm C\) , усі елементи якої дорівнюють попарній різниці
всіх відповідних елементів матриць \(\mathrm A\) і \(\mathrm B\), тобто
фактичноце полементне віднімання.

\begin{pythoncode}
    # Створимо першу матрицю 4x4 з випадковими цілими числами від 1 до 10
    matrix1 = np.random.randint(1, 11, (4, 4))

    # Створимо другу матрицю 4x4 з випадковими цілими числами від 1 до 10
    matrix2 = np.random.randint(1, 11, (4, 4))

    # Виведемо обидві матриці
    print("Перша матриця:")
    print(matrix1)
    print("Друга матриця:")
    print(matrix2)

    # Знайдемо різницю матриць
    result = matrix1 - matrix2

    # Виведемо результат
    print("Різниця матриць:")
    print(result)
\end{pythoncode}

\begin{out}
	Перша матриця:
	[[ 6  4 10  4]
	 [ 6  3  7 10]
	 [ 5  9  6  4]
	 [ 5  9 10  5]]
	Друга матриця:
	[[6 6 4 2]
	 [6 1 2 8]
	 [2 9 1 5]
	 [5 3 5 5]]
	Різниця матриць:
	[[ 0 -2  6  2]
	 [ 0  2  5  2]
	 [ 3  0  5 -1]
	 [ 0  6  5  0]]
\end{out}

\section{Створіть матрицю розміром 3x3, заповніть її випадковими дійсними числами в діапазоні від 0 до 1 та знайдіть вектор-стовпчик, що містить суму елементів кожного рядка матриці.}

\begin{pythoncode}
    # Створимо матрицю 3x3 з випадковими дійсними числами в діапазоні від 0 до 1
    matrix = np.random.rand(3, 3)

    # Виведемо матрицю
    print("Матриця:")
    print(matrix)

    # Знайдемо суму елементів кожного рядка матриці та створимо вектор-стовпчик
    sum_of_rows = np.sum(matrix, axis=1, keepdims=True)

    # Виведемо вектор-стовпчик із сумами
    print("Вектор-стовпчик із сумами рядків:")
    print(sum_of_rows)
\end{pythoncode}

\begin{out}
	Матриця:
	[[0.01297979 0.47161382 0.59052176]
	 [0.55785916 0.35450255 0.41946586]
	 [0.67527101 0.41623059 0.6897199 ]]
	Вектор-стовпчик із сумами рядків:
	[[1.07511538]
	 [1.33182756]
	 [1.78122151]]
\end{out}

\section{Створіть матрицю розміром 3x4 з довільними цілими числами і створінь матрицю з квадратами цих чисел.}

\begin{pythoncode}
# Створимо матрицю 3x4 з довільними цілими числами (від -10 до 10)
matrix = np.random.randint(-10, 11, (3, 4))

# Виведемо початкову матрицю
print("Початкова матриця:")
print(matrix)

# Знайдемо матрицю з квадратами чисел
squared_matrix = np.square(matrix)

# Виведемо матрицю з квадратами чисел
print("Матриця з квадратами чисел:")
print(squared_matrix)
\end{pythoncode}

\begin{out}
	Початкова матриця:
	[[-7 -9 -5  3]
	 [-2 -8 -6 10]
	 [-1  2 -6  9]]
	Матриця з квадратами чисел:
	[[ 49  81  25   9]
	 [  4  64  36 100]
	 [  1   4  36  81]]
\end{out}

\section{Створіть вектор розміром 4, заповніть його випадковими цілими числами в діапазоні від 1 до 50 та знайдіть вектор з квадратними коренями цих чисел}

\begin{pythoncode}
    # Створимо вектор розміром 4 з випадковими цілими числами від 1 до 50
    vector = np.random.randint(1, 51, 4)

    # Виведемо початковий вектор
    print("Початковий вектор:", vector)

    # Знайдемо вектор з квадратними коренями чисел
    sqrt_vector = np.sqrt(vector)

    # Виведемо вектор з квадратними коренями
    print("Вектор з квадратними коренями чисел:", sqrt_vector)
\end{pythoncode}

\begin{out}
	Початковий вектор: [49 12 50 11]
	Вектор з квадратними коренями чисел: [7. 3.46410162 7.07106781 3.31662479]
\end{out}





\end{document}
