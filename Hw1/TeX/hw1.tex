%%============================ Compiler Directives =======================%%
%%                                                                        %%
% !TeX program = pdflatex
% !TeX encoding = utf8
% !TeX spellcheck = uk_UA
%%                                                                        %%
%%============================== Клас документа ==========================%%
%%                                                                        %%
\documentclass[]{article}
\usepackage[fontsize = 12pt]{fontsize}
\usepackage{ifluatex}
%%                                                                        %%
%%========================== Мови, шрифти та кодування ===================%%
%%
\ifluatex                                                                        %%
	\usepackage{fontspec}
	\setsansfont{CMU Sans Serif}%{Arial}
	\setmainfont{CMU Serif}%{Times New Roman}
	\setmonofont{CMU Typewriter Text}%{Consolas}
	\defaultfontfeatures{Ligatures={TeX}}
	\usepackage[math-style=TeX]{unicode-math}
\else
	\usepackage[utf8]{inputenc}
	\usepackage[T2A,T1]{fontenc}
	\usepackage{amsmath}
	%\usepackage{pscyr}
	\usepackage{cmap}
\fi
\usepackage[english, russian, ukrainian]{babel}

    \usepackage[breakable, most, minted]{tcolorbox}
    \usepackage{parskip} % Stop auto-indenting (to mimic markdown behaviour)


    % Basic figure setup, for now with no caption control since it's done
    % automatically by Pandoc (which extracts ![](path) syntax from Markdown).
    \usepackage{graphicx}
    % Maintain compatibility with old templates. Remove in nbconvert 6.0
    \let\Oldincludegraphics\includegraphics
    % Ensure that by default, figures have no caption (until we provide a
    % proper Figure object with a Caption API and a way to capture that
    % in the conversion process - todo).
    \usepackage{caption}
    \DeclareCaptionFormat{nocaption}{}
    \captionsetup{format=nocaption,aboveskip=0pt,belowskip=0pt}

    \usepackage{float}
    \floatplacement{figure}{H} % forces figures to be placed at the correct location
    \usepackage{xcolor} % Allow colors to be defined
    \usepackage{enumerate} % Needed for markdown enumerations to work
    \usepackage{geometry} % Used to adjust the document margins
    \usepackage{amsmath} % Equations
    \usepackage{amssymb} % Equations
    \usepackage{textcomp} % defines textquotesingle
    % Hack from http://tex.stackexchange.com/a/47451/13684:
    \AtBeginDocument{%
        \def\PYZsq{\textquotesingle}% Upright quotes in Pygmentized code
    }
    \usepackage{upquote} % Upright quotes for verbatim code
    \usepackage{eurosym} % defines \euro

    \usepackage{iftex}
    \ifPDFTeX
        \usepackage[T1]{fontenc}
        \IfFileExists{alphabeta.sty}{
              \usepackage{alphabeta}
          }{
              \usepackage[mathletters]{ucs}
              \usepackage[utf8x]{inputenc}
          }
    \else
        \usepackage{fontspec}
        \usepackage{unicode-math}
    \fi

    \usepackage{fancyvrb} % verbatim replacement that allows latex
    \usepackage{grffile} % extends the file name processing of package graphics
                         % to support a larger range
    \makeatletter % fix for old versions of grffile with XeLaTeX
    \@ifpackagelater{grffile}{2019/11/01}
    {
      % Do nothing on new versions
    }
    {
      \def\Gread@@xetex#1{%
        \IfFileExists{"\Gin@base".bb}%
        {\Gread@eps{\Gin@base.bb}}%
        {\Gread@@xetex@aux#1}%
      }
    }
    \makeatother
    \usepackage[Export]{adjustbox} % Used to constrain images to a maximum size
    \adjustboxset{max size={0.9\linewidth}{0.9\paperheight}}

    % The hyperref package gives us a pdf with properly built
    % internal navigation ('pdf bookmarks' for the table of contents,
    % internal cross-reference links, web links for URLs, etc.)
    \usepackage{hyperref}
    % The default LaTeX title has an obnoxious amount of whitespace. By default,
    % titling removes some of it. It also provides customization options.
    \usepackage{titling}
    \usepackage{longtable} % longtable support required by pandoc >1.10
    \usepackage{booktabs}  % table support for pandoc > 1.12.2
    \usepackage{array}     % table support for pandoc >= 2.11.3
    \usepackage{calc}      % table minipage width calculation for pandoc >= 2.11.1
    \usepackage[inline]{enumitem} % IRkernel/repr support (it uses the enumerate* environment)
    \usepackage[normalem]{ulem} % ulem is needed to support strikethroughs (\sout)
                                % normalem makes italics be italics, not underlines
    \usepackage{mathrsfs}



    % Colors for the hyperref package
    \definecolor{urlcolor}{rgb}{0,.145,.698}
    \definecolor{linkcolor}{rgb}{.71,0.21,0.01}
    \definecolor{citecolor}{rgb}{.12,.54,.11}

    % ANSI colors
    \definecolor{ansi-black}{HTML}{3E424D}
    \definecolor{ansi-black-intense}{HTML}{282C36}
    \definecolor{ansi-red}{HTML}{E75C58}
    \definecolor{ansi-red-intense}{HTML}{B22B31}
    \definecolor{ansi-green}{HTML}{00A250}
    \definecolor{ansi-green-intense}{HTML}{007427}
    \definecolor{ansi-yellow}{HTML}{DDB62B}
    \definecolor{ansi-yellow-intense}{HTML}{B27D12}
    \definecolor{ansi-blue}{HTML}{208FFB}
    \definecolor{ansi-blue-intense}{HTML}{0065CA}
    \definecolor{ansi-magenta}{HTML}{D160C4}
    \definecolor{ansi-magenta-intense}{HTML}{A03196}
    \definecolor{ansi-cyan}{HTML}{60C6C8}
    \definecolor{ansi-cyan-intense}{HTML}{258F8F}
    \definecolor{ansi-white}{HTML}{C5C1B4}
    \definecolor{ansi-white-intense}{HTML}{A1A6B2}
    \definecolor{ansi-default-inverse-fg}{HTML}{FFFFFF}
    \definecolor{ansi-default-inverse-bg}{HTML}{000000}

    % common color for the border for error outputs.
    \definecolor{outerrorbackground}{HTML}{FFDFDF}

    % commands and environments needed by pandoc snippets
    % extracted from the output of `pandoc -s`
    \providecommand{\tightlist}{%
      \setlength{\itemsep}{0pt}\setlength{\parskip}{0pt}}
    \DefineVerbatimEnvironment{Highlighting}{Verbatim}{commandchars=\\\{\}}
    % Add ',fontsize=\small' for more characters per line
    \newenvironment{Shaded}{}{}
    \newcommand{\KeywordTok}[1]{\textcolor[rgb]{0.00,0.44,0.13}{\textbf{{#1}}}}
    \newcommand{\DataTypeTok}[1]{\textcolor[rgb]{0.56,0.13,0.00}{{#1}}}
    \newcommand{\DecValTok}[1]{\textcolor[rgb]{0.25,0.63,0.44}{{#1}}}
    \newcommand{\BaseNTok}[1]{\textcolor[rgb]{0.25,0.63,0.44}{{#1}}}
    \newcommand{\FloatTok}[1]{\textcolor[rgb]{0.25,0.63,0.44}{{#1}}}
    \newcommand{\CharTok}[1]{\textcolor[rgb]{0.25,0.44,0.63}{{#1}}}
    \newcommand{\StringTok}[1]{\textcolor[rgb]{0.25,0.44,0.63}{{#1}}}
    \newcommand{\CommentTok}[1]{\textcolor[rgb]{0.38,0.63,0.69}{\textit{{#1}}}}
    \newcommand{\OtherTok}[1]{\textcolor[rgb]{0.00,0.44,0.13}{{#1}}}
    \newcommand{\AlertTok}[1]{\textcolor[rgb]{1.00,0.00,0.00}{\textbf{{#1}}}}
    \newcommand{\FunctionTok}[1]{\textcolor[rgb]{0.02,0.16,0.49}{{#1}}}
    \newcommand{\RegionMarkerTok}[1]{{#1}}
    \newcommand{\ErrorTok}[1]{\textcolor[rgb]{1.00,0.00,0.00}{\textbf{{#1}}}}
    \newcommand{\NormalTok}[1]{{#1}}

    % Additional commands for more recent versions of Pandoc
    \newcommand{\ConstantTok}[1]{\textcolor[rgb]{0.53,0.00,0.00}{{#1}}}
    \newcommand{\SpecialCharTok}[1]{\textcolor[rgb]{0.25,0.44,0.63}{{#1}}}
    \newcommand{\VerbatimStringTok}[1]{\textcolor[rgb]{0.25,0.44,0.63}{{#1}}}
    \newcommand{\SpecialStringTok}[1]{\textcolor[rgb]{0.73,0.40,0.53}{{#1}}}
    \newcommand{\ImportTok}[1]{{#1}}
    \newcommand{\DocumentationTok}[1]{\textcolor[rgb]{0.73,0.13,0.13}{\textit{{#1}}}}
    \newcommand{\AnnotationTok}[1]{\textcolor[rgb]{0.38,0.63,0.69}{\textbf{\textit{{#1}}}}}
    \newcommand{\CommentVarTok}[1]{\textcolor[rgb]{0.38,0.63,0.69}{\textbf{\textit{{#1}}}}}
    \newcommand{\VariableTok}[1]{\textcolor[rgb]{0.10,0.09,0.49}{{#1}}}
    \newcommand{\ControlFlowTok}[1]{\textcolor[rgb]{0.00,0.44,0.13}{\textbf{{#1}}}}
    \newcommand{\OperatorTok}[1]{\textcolor[rgb]{0.40,0.40,0.40}{{#1}}}
    \newcommand{\BuiltInTok}[1]{{#1}}
    \newcommand{\ExtensionTok}[1]{{#1}}
    \newcommand{\PreprocessorTok}[1]{\textcolor[rgb]{0.74,0.48,0.00}{{#1}}}
    \newcommand{\AttributeTok}[1]{\textcolor[rgb]{0.49,0.56,0.16}{{#1}}}
    \newcommand{\InformationTok}[1]{\textcolor[rgb]{0.38,0.63,0.69}{\textbf{\textit{{#1}}}}}
    \newcommand{\WarningTok}[1]{\textcolor[rgb]{0.38,0.63,0.69}{\textbf{\textit{{#1}}}}}


    % Define a nice break command that doesn't care if a line doesn't already
    % exist.
    \def\br{\hspace*{\fill} \\* }
    % Math Jax compatibility definitions
    \def\gt{>}
    \def\lt{<}
    \let\Oldtex\TeX
    \let\Oldlatex\LaTeX
    \renewcommand{\TeX}{\textrm{\Oldtex}}
    \renewcommand{\LaTeX}{\textrm{\Oldlatex}}
    % Document parameters
    % Document title
    \title{Домашнє завдання №1 (Модуль 1. Знайомство з Data Science)}





% Pygments definitions
\makeatletter
\def\PY@reset{\let\PY@it=\relax \let\PY@bf=\relax%
    \let\PY@ul=\relax \let\PY@tc=\relax%
    \let\PY@bc=\relax \let\PY@ff=\relax}
\def\PY@tok#1{\csname PY@tok@#1\endcsname}
\def\PY@toks#1+{\ifx\relax#1\empty\else%
    \PY@tok{#1}\expandafter\PY@toks\fi}
\def\PY@do#1{\PY@bc{\PY@tc{\PY@ul{%
    \PY@it{\PY@bf{\PY@ff{#1}}}}}}}
\def\PY#1#2{\PY@reset\PY@toks#1+\relax+\PY@do{#2}}

\@namedef{PY@tok@w}{\def\PY@tc##1{\textcolor[rgb]{0.73,0.73,0.73}{##1}}}
\@namedef{PY@tok@c}{\let\PY@it=\textit\def\PY@tc##1{\textcolor[rgb]{0.24,0.48,0.48}{##1}}}
\@namedef{PY@tok@cp}{\def\PY@tc##1{\textcolor[rgb]{0.61,0.40,0.00}{##1}}}
\@namedef{PY@tok@k}{\let\PY@bf=\textbf\def\PY@tc##1{\textcolor[rgb]{0.00,0.50,0.00}{##1}}}
\@namedef{PY@tok@kp}{\def\PY@tc##1{\textcolor[rgb]{0.00,0.50,0.00}{##1}}}
\@namedef{PY@tok@kt}{\def\PY@tc##1{\textcolor[rgb]{0.69,0.00,0.25}{##1}}}
\@namedef{PY@tok@o}{\def\PY@tc##1{\textcolor[rgb]{0.40,0.40,0.40}{##1}}}
\@namedef{PY@tok@ow}{\let\PY@bf=\textbf\def\PY@tc##1{\textcolor[rgb]{0.67,0.13,1.00}{##1}}}
\@namedef{PY@tok@nb}{\def\PY@tc##1{\textcolor[rgb]{0.00,0.50,0.00}{##1}}}
\@namedef{PY@tok@nf}{\def\PY@tc##1{\textcolor[rgb]{0.00,0.00,1.00}{##1}}}
\@namedef{PY@tok@nc}{\let\PY@bf=\textbf\def\PY@tc##1{\textcolor[rgb]{0.00,0.00,1.00}{##1}}}
\@namedef{PY@tok@nn}{\let\PY@bf=\textbf\def\PY@tc##1{\textcolor[rgb]{0.00,0.00,1.00}{##1}}}
\@namedef{PY@tok@ne}{\let\PY@bf=\textbf\def\PY@tc##1{\textcolor[rgb]{0.80,0.25,0.22}{##1}}}
\@namedef{PY@tok@nv}{\def\PY@tc##1{\textcolor[rgb]{0.10,0.09,0.49}{##1}}}
\@namedef{PY@tok@no}{\def\PY@tc##1{\textcolor[rgb]{0.53,0.00,0.00}{##1}}}
\@namedef{PY@tok@nl}{\def\PY@tc##1{\textcolor[rgb]{0.46,0.46,0.00}{##1}}}
\@namedef{PY@tok@ni}{\let\PY@bf=\textbf\def\PY@tc##1{\textcolor[rgb]{0.44,0.44,0.44}{##1}}}
\@namedef{PY@tok@na}{\def\PY@tc##1{\textcolor[rgb]{0.41,0.47,0.13}{##1}}}
\@namedef{PY@tok@nt}{\let\PY@bf=\textbf\def\PY@tc##1{\textcolor[rgb]{0.00,0.50,0.00}{##1}}}
\@namedef{PY@tok@nd}{\def\PY@tc##1{\textcolor[rgb]{0.67,0.13,1.00}{##1}}}
\@namedef{PY@tok@s}{\def\PY@tc##1{\textcolor[rgb]{0.73,0.13,0.13}{##1}}}
\@namedef{PY@tok@sd}{\let\PY@it=\textit\def\PY@tc##1{\textcolor[rgb]{0.73,0.13,0.13}{##1}}}
\@namedef{PY@tok@si}{\let\PY@bf=\textbf\def\PY@tc##1{\textcolor[rgb]{0.64,0.35,0.47}{##1}}}
\@namedef{PY@tok@se}{\let\PY@bf=\textbf\def\PY@tc##1{\textcolor[rgb]{0.67,0.36,0.12}{##1}}}
\@namedef{PY@tok@sr}{\def\PY@tc##1{\textcolor[rgb]{0.64,0.35,0.47}{##1}}}
\@namedef{PY@tok@ss}{\def\PY@tc##1{\textcolor[rgb]{0.10,0.09,0.49}{##1}}}
\@namedef{PY@tok@sx}{\def\PY@tc##1{\textcolor[rgb]{0.00,0.50,0.00}{##1}}}
\@namedef{PY@tok@m}{\def\PY@tc##1{\textcolor[rgb]{0.40,0.40,0.40}{##1}}}
\@namedef{PY@tok@gh}{\let\PY@bf=\textbf\def\PY@tc##1{\textcolor[rgb]{0.00,0.00,0.50}{##1}}}
\@namedef{PY@tok@gu}{\let\PY@bf=\textbf\def\PY@tc##1{\textcolor[rgb]{0.50,0.00,0.50}{##1}}}
\@namedef{PY@tok@gd}{\def\PY@tc##1{\textcolor[rgb]{0.63,0.00,0.00}{##1}}}
\@namedef{PY@tok@gi}{\def\PY@tc##1{\textcolor[rgb]{0.00,0.52,0.00}{##1}}}
\@namedef{PY@tok@gr}{\def\PY@tc##1{\textcolor[rgb]{0.89,0.00,0.00}{##1}}}
\@namedef{PY@tok@ge}{\let\PY@it=\textit}
\@namedef{PY@tok@gs}{\let\PY@bf=\textbf}
\@namedef{PY@tok@gp}{\let\PY@bf=\textbf\def\PY@tc##1{\textcolor[rgb]{0.00,0.00,0.50}{##1}}}
\@namedef{PY@tok@go}{\def\PY@tc##1{\textcolor[rgb]{0.44,0.44,0.44}{##1}}}
\@namedef{PY@tok@gt}{\def\PY@tc##1{\textcolor[rgb]{0.00,0.27,0.87}{##1}}}
\@namedef{PY@tok@err}{\def\PY@bc##1{{\setlength{\fboxsep}{\string -\fboxrule}\fcolorbox[rgb]{1.00,0.00,0.00}{1,1,1}{\strut ##1}}}}
\@namedef{PY@tok@kc}{\let\PY@bf=\textbf\def\PY@tc##1{\textcolor[rgb]{0.00,0.50,0.00}{##1}}}
\@namedef{PY@tok@kd}{\let\PY@bf=\textbf\def\PY@tc##1{\textcolor[rgb]{0.00,0.50,0.00}{##1}}}
\@namedef{PY@tok@kn}{\let\PY@bf=\textbf\def\PY@tc##1{\textcolor[rgb]{0.00,0.50,0.00}{##1}}}
\@namedef{PY@tok@kr}{\let\PY@bf=\textbf\def\PY@tc##1{\textcolor[rgb]{0.00,0.50,0.00}{##1}}}
\@namedef{PY@tok@bp}{\def\PY@tc##1{\textcolor[rgb]{0.00,0.50,0.00}{##1}}}
\@namedef{PY@tok@fm}{\def\PY@tc##1{\textcolor[rgb]{0.00,0.00,1.00}{##1}}}
\@namedef{PY@tok@vc}{\def\PY@tc##1{\textcolor[rgb]{0.10,0.09,0.49}{##1}}}
\@namedef{PY@tok@vg}{\def\PY@tc##1{\textcolor[rgb]{0.10,0.09,0.49}{##1}}}
\@namedef{PY@tok@vi}{\def\PY@tc##1{\textcolor[rgb]{0.10,0.09,0.49}{##1}}}
\@namedef{PY@tok@vm}{\def\PY@tc##1{\textcolor[rgb]{0.10,0.09,0.49}{##1}}}
\@namedef{PY@tok@sa}{\def\PY@tc##1{\textcolor[rgb]{0.73,0.13,0.13}{##1}}}
\@namedef{PY@tok@sb}{\def\PY@tc##1{\textcolor[rgb]{0.73,0.13,0.13}{##1}}}
\@namedef{PY@tok@sc}{\def\PY@tc##1{\textcolor[rgb]{0.73,0.13,0.13}{##1}}}
\@namedef{PY@tok@dl}{\def\PY@tc##1{\textcolor[rgb]{0.73,0.13,0.13}{##1}}}
\@namedef{PY@tok@s2}{\def\PY@tc##1{\textcolor[rgb]{0.73,0.13,0.13}{##1}}}
\@namedef{PY@tok@sh}{\def\PY@tc##1{\textcolor[rgb]{0.73,0.13,0.13}{##1}}}
\@namedef{PY@tok@s1}{\def\PY@tc##1{\textcolor[rgb]{0.73,0.13,0.13}{##1}}}
\@namedef{PY@tok@mb}{\def\PY@tc##1{\textcolor[rgb]{0.40,0.40,0.40}{##1}}}
\@namedef{PY@tok@mf}{\def\PY@tc##1{\textcolor[rgb]{0.40,0.40,0.40}{##1}}}
\@namedef{PY@tok@mh}{\def\PY@tc##1{\textcolor[rgb]{0.40,0.40,0.40}{##1}}}
\@namedef{PY@tok@mi}{\def\PY@tc##1{\textcolor[rgb]{0.40,0.40,0.40}{##1}}}
\@namedef{PY@tok@il}{\def\PY@tc##1{\textcolor[rgb]{0.40,0.40,0.40}{##1}}}
\@namedef{PY@tok@mo}{\def\PY@tc##1{\textcolor[rgb]{0.40,0.40,0.40}{##1}}}
\@namedef{PY@tok@ch}{\let\PY@it=\textit\def\PY@tc##1{\textcolor[rgb]{0.24,0.48,0.48}{##1}}}
\@namedef{PY@tok@cm}{\let\PY@it=\textit\def\PY@tc##1{\textcolor[rgb]{0.24,0.48,0.48}{##1}}}
\@namedef{PY@tok@cpf}{\let\PY@it=\textit\def\PY@tc##1{\textcolor[rgb]{0.24,0.48,0.48}{##1}}}
\@namedef{PY@tok@c1}{\let\PY@it=\textit\def\PY@tc##1{\textcolor[rgb]{0.24,0.48,0.48}{##1}}}
\@namedef{PY@tok@cs}{\let\PY@it=\textit\def\PY@tc##1{\textcolor[rgb]{0.24,0.48,0.48}{##1}}}

\def\PYZbs{\char`\\}
\def\PYZus{\char`\_}
\def\PYZob{\char`\{}
\def\PYZcb{\char`\}}
\def\PYZca{\char`\^}
\def\PYZam{\char`\&}
\def\PYZlt{\char`\<}
\def\PYZgt{\char`\>}
\def\PYZsh{\char`\#}
\def\PYZpc{\char`\%}
\def\PYZdl{\char`$}
\def\PYZhy{\char`\-}
\def\PYZsq{\char`\'}
\def\PYZdq{\char`\"}
\def\PYZti{\char`\~}
% for compatibility with earlier versions
\def\PYZat{@}
\def\PYZlb{[}
\def\PYZrb{]}
\makeatother


    % For linebreaks inside Verbatim environment from package fancyvrb.
    \makeatletter
        \newbox\Wrappedcontinuationbox
        \newbox\Wrappedvisiblespacebox
        \newcommand*\Wrappedvisiblespace {\textcolor{red}{\textvisiblespace}}
        \newcommand*\Wrappedcontinuationsymbol {\textcolor{red}{\llap{\tiny$\m@th\hookrightarrow$}}}
        \newcommand*\Wrappedcontinuationindent {3ex }
        \newcommand*\Wrappedafterbreak {\kern\Wrappedcontinuationindent\copy\Wrappedcontinuationbox}
        % Take advantage of the already applied Pygments mark-up to insert
        % potential linebreaks for TeX processing.
        %        {, <, #, %, $, ' and ": go to next line.
        %        _, }, ^, &, >, - and ~: stay at end of broken line.
        % Use of \textquotesingle for straight quote.
        \newcommand*\Wrappedbreaksatspecials {%
            \def\PYGZus{\discretionary{\char`\_}{\Wrappedafterbreak}{\char`\_}}%
            \def\PYGZob{\discretionary{}{\Wrappedafterbreak\char`\{}{\char`\{}}%
            \def\PYGZcb{\discretionary{\char`\}}{\Wrappedafterbreak}{\char`\}}}%
            \def\PYGZca{\discretionary{\char`\^}{\Wrappedafterbreak}{\char`\^}}%
            \def\PYGZam{\discretionary{\char`\&}{\Wrappedafterbreak}{\char`\&}}%
            \def\PYGZlt{\discretionary{}{\Wrappedafterbreak\char`\<}{\char`\<}}%
            \def\PYGZgt{\discretionary{\char`\>}{\Wrappedafterbreak}{\char`\>}}%
            \def\PYGZsh{\discretionary{}{\Wrappedafterbreak\char`\#}{\char`\#}}%
            \def\PYGZpc{\discretionary{}{\Wrappedafterbreak\char`\%}{\char`\%}}%
            \def\PYGZdl{\discretionary{}{\Wrappedafterbreak\char`$}{\char`$}}%
            \def\PYGZhy{\discretionary{\char`\-}{\Wrappedafterbreak}{\char`\-}}%
            \def\PYGZsq{\discretionary{}{\Wrappedafterbreak\textquotesingle}{\textquotesingle}}%
            \def\PYGZdq{\discretionary{}{\Wrappedafterbreak\char`\"}{\char`\"}}%
            \def\PYGZti{\discretionary{\char`\~}{\Wrappedafterbreak}{\char`\~}}%
        }
        % Some characters . , ; ? ! / are not pygmentized.
        % This macro makes them "active" and they will insert potential linebreaks
        \newcommand*\Wrappedbreaksatpunct {%
            \lccode`\~`\.\lowercase{\def~}{\discretionary{\hbox{\char`\.}}{\Wrappedafterbreak}{\hbox{\char`\.}}}%
            \lccode`\~`\,\lowercase{\def~}{\discretionary{\hbox{\char`\,}}{\Wrappedafterbreak}{\hbox{\char`\,}}}%
            \lccode`\~`\;\lowercase{\def~}{\discretionary{\hbox{\char`\;}}{\Wrappedafterbreak}{\hbox{\char`\;}}}%
            \lccode`\~`\:\lowercase{\def~}{\discretionary{\hbox{\char`\:}}{\Wrappedafterbreak}{\hbox{\char`\:}}}%
            \lccode`\~`\?\lowercase{\def~}{\discretionary{\hbox{\char`\?}}{\Wrappedafterbreak}{\hbox{\char`\?}}}%
            \lccode`\~`\!\lowercase{\def~}{\discretionary{\hbox{\char`\!}}{\Wrappedafterbreak}{\hbox{\char`\!}}}%
            \lccode`\~`\/\lowercase{\def~}{\discretionary{\hbox{\char`\/}}{\Wrappedafterbreak}{\hbox{\char`\/}}}%
            \catcode`\.\active
            \catcode`\,\active
            \catcode`\;\active
            \catcode`\:\active
            \catcode`\?\active
            \catcode`\!\active
            \catcode`\/\active
            \lccode`\~`\~
        }
    \makeatother

    \let\OriginalVerbatim=\Verbatim
    \makeatletter
    \renewcommand{\Verbatim}[1][1]{%
        %\parskip\z@skip
        \sbox\Wrappedcontinuationbox {\Wrappedcontinuationsymbol}%
        \sbox\Wrappedvisiblespacebox {\FV@SetupFont\Wrappedvisiblespace}%
        \def\FancyVerbFormatLine ##1{\hsize\linewidth
            \vtop{\raggedright\hyphenpenalty\z@\exhyphenpenalty\z@
                \doublehyphendemerits\z@\finalhyphendemerits\z@
                \strut ##1\strut}%
        }%
        % If the linebreak is at a space, the latter will be displayed as visible
        % space at end of first line, and a continuation symbol starts next line.
        % Stretch/shrink are however usually zero for typewriter font.
        \def\FV@Space {%
            \nobreak\hskip\z@ plus\fontdimen3\font minus\fontdimen4\font
            \discretionary{\copy\Wrappedvisiblespacebox}{\Wrappedafterbreak}
            {\kern\fontdimen2\font}%
        }%

        % Allow breaks at special characters using \PYG... macros.
        \Wrappedbreaksatspecials
        % Breaks at punctuation characters . , ; ? ! and / need catcode=\active
        \OriginalVerbatim[#1,codes*=\Wrappedbreaksatpunct]%
    }
    \makeatother

    % Exact colors from NB
    \definecolor{incolor}{HTML}{303F9F}
    \definecolor{outcolor}{HTML}{D84315}
    \definecolor{cellborder}{HTML}{CFCFCF}
    \definecolor{cellbackground}{HTML}{F7F7F7}

    % prompt
    \makeatletter
    \newcommand{\boxspacing}{\kern\kvtcb@left@rule\kern\kvtcb@boxsep}
    \makeatother
    \newcommand{\prompt}[4]{
        {\ttfamily\llap{{\color{#2}[#3]:\hspace{3pt}#4}}\vspace{-\baselineskip}}
    }



    % Prevent overflowing lines due to hard-to-break entities
    \sloppy
    % Setup hyperref package
    \hypersetup{
      breaklinks=true,  % so long urls are correctly broken across lines
      colorlinks=true,
      urlcolor=urlcolor,
      linkcolor=linkcolor,
      citecolor=citecolor,
      }
    % Slightly bigger margins than the latex defaults

    \geometry{verbose,tmargin=1in,bmargin=1in,lmargin=1in,rmargin=1in}

\newcounter{in}
\def\in{\prompt{In}{incolor}{\stepcounter{in}\thein}{\boxspacing}}
\usepackage{minted}

\newtcblisting{pythoncode}{
    listing engine=minted,
    listing only,
    breakable,
    size=fbox,
    boxrule=1pt,
    pad at break*=1mm,
    colback=cellbackground,
    colframe=cellborder,
    minted language=python,
    minted options={
        autogobble,
        fontsize=\small,
        breaklines
    }
}

\begin{document}

\maketitle





\section{Імпорт модулів}

\in%
\begin{pythoncode}
import numpy as np
\end{pythoncode}




\section{Створіть одновимірний масив (вектор) з першими 10-ма
	натуральними числами та виведіть його значення.}

Натуральні числа --- це числа, що виникають природним чином при лічбі.
Це числа: \(1\), \(2\), \(3\), \(4\), \(\ldots\) Множину натуральних
чисел прийнято позначати \(\mathbb{N}\).

\in
\begin{pythoncode}
vector = np.arange(1, 11)
print(vector)
\end{pythoncode}


\begin{Verbatim}[commandchars=\\\{\}]
	[ 1  2  3  4  5  6  7  8  9 10]
\end{Verbatim}

\section{Створіть двовимірний масив (матрицю) розміром 3x3,
	заповніть його нулями та виведіть його
	значення.}
\in
\begin{pythoncode}
matrix = np.zeros((3, 3))
print(matrix)
\end{pythoncode}

\begin{Verbatim}[commandchars=\\\{\}]
[[0. 0. 0.]
 [0. 0. 0.]
 [0. 0. 0.]]
\end{Verbatim}

\section{\texorpdfstring{Створіть масив розміром \(5\times 5\), 		заповніть його випадковими цілими числами в діапазоні від 1 до 10 та 		виведіть його
		значення.}{3. Створіть масив розміром 5\textbackslash times 5, заповніть його випадковими цілими числами в діапазоні від 1 до 10 та виведіть його значення.}}

\begin{tcolorbox}[breakable, size=fbox, boxrule=1pt, pad at break*=1mm,colback=cellbackground, colframe=cellborder]
	\prompt{In}{incolor}{5}{\boxspacing}
	\begin{Verbatim}[commandchars=\\\{\}]
		\PY{n}{random\PYZus{}matrix} \PY{o}{=} \PY{n}{np}\PY{o}{.}\PY{n}{random}\PY{o}{.}\PY{n}{randint}\PY{p}{(}\PY{l+m+mi}{1}\PY{p}{,} \PY{l+m+mi}{11}\PY{p}{,} \PY{p}{(}\PY{l+m+mi}{5}\PY{p}{,} \PY{l+m+mi}{5}\PY{p}{)}\PY{p}{)}
		\PY{n+nb}{print}\PY{p}{(}\PY{n}{random\PYZus{}matrix}\PY{p}{)}
	\end{Verbatim}
\end{tcolorbox}

\begin{Verbatim}[commandchars=\\\{\}]
	[[ 7 10  4  6  7]
		[ 1 10  1  1 10]
		[10  9 10  6  3]
		[10  7  9  6  6]
		[ 9  2  4  4  4]]
\end{Verbatim}

\section{\texorpdfstring{Створіть масив розміром \(4\times 4\),
		заповніть його випадковими дійсними числами в діапазоні від 0 до 1 та
		виведіть його
		значення.}{Створіть масив розміром 4\textbackslash times 4, заповніть його випадковими дійсними числами в діапазоні від 0 до 1 та виведіть його значення.}}

\begin{tcolorbox}[breakable, size=fbox, boxrule=1pt, pad at break*=1mm,colback=cellbackground, colframe=cellborder]
	\prompt{In}{incolor}{6}{\boxspacing}
	\begin{Verbatim}[commandchars=\\\{\}]
		\PY{n}{random\PYZus{}matrix} \PY{o}{=} \PY{n}{np}\PY{o}{.}\PY{n}{random}\PY{o}{.}\PY{n}{rand}\PY{p}{(}\PY{l+m+mi}{4}\PY{p}{,} \PY{l+m+mi}{4}\PY{p}{)}
		\PY{n+nb}{print}\PY{p}{(}\PY{n}{random\PYZus{}matrix}\PY{p}{)}
	\end{Verbatim}
\end{tcolorbox}

\begin{Verbatim}[commandchars=\\\{\}]
	[[0.34716231 0.38666569 0.60847232 0.44947206]
		[0.28441679 0.24889235 0.69068513 0.44390396]
		[0.45757561 0.30179135 0.30334876 0.14319334]
		[0.19297175 0.18552924 0.3236188  0.56707376]]
\end{Verbatim}

\section{Створіть два одновимірних масиви розміром 5, заповніть їх
	випадковими цілими числами в діапазоні від 1 до 10 та виконайте на них
	поелементні операції додавання, віднімання та
	множення.}

\subsection{Створення
	масивів}

\begin{tcolorbox}[breakable, size=fbox, boxrule=1pt, pad at break*=1mm,colback=cellbackground, colframe=cellborder]
	\prompt{In}{incolor}{7}{\boxspacing}
	\begin{Verbatim}[commandchars=\\\{\}]
		\PY{n}{array1} \PY{o}{=} \PY{n}{np}\PY{o}{.}\PY{n}{random}\PY{o}{.}\PY{n}{randint}\PY{p}{(}\PY{l+m+mi}{1}\PY{p}{,} \PY{l+m+mi}{11}\PY{p}{,} \PY{l+m+mi}{5}\PY{p}{)}
		\PY{n}{array2} \PY{o}{=} \PY{n}{np}\PY{o}{.}\PY{n}{random}\PY{o}{.}\PY{n}{randint}\PY{p}{(}\PY{l+m+mi}{1}\PY{p}{,} \PY{l+m+mi}{11}\PY{p}{,} \PY{l+m+mi}{5}\PY{p}{)}

		\PY{n+nb}{print}\PY{p}{(}\PY{l+s+s2}{\PYZdq{}}\PY{l+s+s2}{Масив 1:}\PY{l+s+s2}{\PYZdq{}}\PY{p}{,} \PY{n}{array1}\PY{p}{)}
		\PY{n+nb}{print}\PY{p}{(}\PY{l+s+s2}{\PYZdq{}}\PY{l+s+s2}{Масив 2:}\PY{l+s+s2}{\PYZdq{}}\PY{p}{,} \PY{n}{array2}\PY{p}{)}
	\end{Verbatim}
\end{tcolorbox}

\begin{Verbatim}[commandchars=\\\{\}]
	Масив 1: [1 5 1 9 9]
	Масив 2: [8 2 4 6 7]
\end{Verbatim}

\subsection{Поелементні
	операції}

\begin{tcolorbox}[breakable, size=fbox, boxrule=1pt, pad at break*=1mm,colback=cellbackground, colframe=cellborder]
	\prompt{In}{incolor}{8}{\boxspacing}
	\begin{Verbatim}[commandchars=\\\{\}]
		\PY{c+c1}{\PYZsh{} Виконаємо поелементну операцію додавання}
		\PY{n}{addition\PYZus{}result} \PY{o}{=} \PY{n}{array1} \PY{o}{+} \PY{n}{array2}
		\PY{n+nb}{print}\PY{p}{(}\PY{l+s+s2}{\PYZdq{}}\PY{l+s+s2}{Результат додавання:}\PY{l+s+s2}{\PYZdq{}}\PY{p}{,} \PY{n}{addition\PYZus{}result}\PY{p}{)}

		\PY{c+c1}{\PYZsh{} Виконаємо поелементну операцію віднімання}
		\PY{n}{subtraction\PYZus{}result} \PY{o}{=} \PY{n}{array1} \PY{o}{\PYZhy{}} \PY{n}{array2}
		\PY{n+nb}{print}\PY{p}{(}\PY{l+s+s2}{\PYZdq{}}\PY{l+s+s2}{Результат віднімання:}\PY{l+s+s2}{\PYZdq{}}\PY{p}{,} \PY{n}{subtraction\PYZus{}result}\PY{p}{)}

		\PY{c+c1}{\PYZsh{} Виконаємо поелементну операцію множення}
		\PY{n}{multiplication\PYZus{}result} \PY{o}{=} \PY{n}{array1} \PY{o}{*} \PY{n}{array2}
		\PY{n+nb}{print}\PY{p}{(}\PY{l+s+s2}{\PYZdq{}}\PY{l+s+s2}{Результат множення:}\PY{l+s+s2}{\PYZdq{}}\PY{p}{,} \PY{n}{multiplication\PYZus{}result}\PY{p}{)}
	\end{Verbatim}
\end{tcolorbox}

\begin{Verbatim}[commandchars=\\\{\}]
	Результат додавання: [ 9  7  5 15 16]
	Результат віднімання: [-7  3 -3  3  2]
	Результат множення: [ 8 10  4 54 63]
\end{Verbatim}

\section{Створіть два вектори розміром 7, заповніть довільними
	числами та знайдіть їх скалярний
	добуток.}

Скалярний добуток (scalar or dot product) можна обчислити за наступною
формулою
\(\vec v \cdot \vec u = v^1 u_1 + v^2 u_2 + v^3 u_3 + \ldots + v^n u_n\)

\begin{tcolorbox}[breakable, size=fbox, boxrule=1pt, pad at break*=1mm,colback=cellbackground, colframe=cellborder]
	\prompt{In}{incolor}{9}{\boxspacing}
	\begin{Verbatim}[commandchars=\\\{\}]
		\PY{n}{vector1} \PY{o}{=} \PY{n}{np}\PY{o}{.}\PY{n}{random}\PY{o}{.}\PY{n}{random}\PY{p}{(}\PY{l+m+mi}{7}\PY{p}{)}
		\PY{n}{vector2} \PY{o}{=} \PY{n}{np}\PY{o}{.}\PY{n}{random}\PY{o}{.}\PY{n}{random}\PY{p}{(}\PY{l+m+mi}{7}\PY{p}{)}

		\PY{n}{scalar\PYZus{}product} \PY{o}{=} \PY{n}{np}\PY{o}{.}\PY{n}{dot}\PY{p}{(}\PY{n}{vector1}\PY{p}{,} \PY{n}{vector2}\PY{p}{)}

		\PY{n+nb}{print}\PY{p}{(}\PY{l+s+s2}{\PYZdq{}}\PY{l+s+s2}{Скалярний добуток векторів:}\PY{l+s+s2}{\PYZdq{}}\PY{p}{,} \PY{n}{scalar\PYZus{}product}\PY{p}{)}
	\end{Verbatim}
\end{tcolorbox}

\begin{Verbatim}[commandchars=\\\{\}]
	Скалярний добуток векторів: 1.8680632685208063
\end{Verbatim}

\section{\texorpdfstring{Створіть дві матриці розміром
		\(2\times2\) та \(2\times3\), заповніть їх випадковими цілими числами в
		діапазоні від \(1\) до \(10\) та перемножте їх між
		собою.}{Створіть дві матриці розміром 2\textbackslash times2 та 2\textbackslash times3, заповніть їх випадковими цілими числами в діапазоні від 1 до 10 та перемножте їх між собою.}}

\begin{tcolorbox}[breakable, size=fbox, boxrule=1pt, pad at break*=1mm,colback=cellbackground, colframe=cellborder]
	\prompt{In}{incolor}{10}{\boxspacing}
	\begin{Verbatim}[commandchars=\\\{\}]
		\PY{n}{matrix1} \PY{o}{=} \PY{n}{np}\PY{o}{.}\PY{n}{random}\PY{o}{.}\PY{n}{randint}\PY{p}{(}\PY{l+m+mi}{1}\PY{p}{,} \PY{l+m+mi}{11}\PY{p}{,} \PY{p}{(}\PY{l+m+mi}{2}\PY{p}{,} \PY{l+m+mi}{2}\PY{p}{)}\PY{p}{)}
		\PY{n}{matrix2} \PY{o}{=} \PY{n}{np}\PY{o}{.}\PY{n}{random}\PY{o}{.}\PY{n}{randint}\PY{p}{(}\PY{l+m+mi}{1}\PY{p}{,} \PY{l+m+mi}{11}\PY{p}{,} \PY{p}{(}\PY{l+m+mi}{2}\PY{p}{,} \PY{l+m+mi}{3}\PY{p}{)}\PY{p}{)}

		\PY{n+nb}{print}\PY{p}{(}\PY{l+s+s2}{\PYZdq{}}\PY{l+s+s2}{Матриця 1:}\PY{l+s+s2}{\PYZdq{}}\PY{p}{)}
		\PY{n+nb}{print}\PY{p}{(}\PY{n}{matrix1}\PY{p}{)}
		\PY{n+nb}{print}\PY{p}{(}\PY{l+s+s2}{\PYZdq{}}\PY{l+s+s2}{Матриця 2:}\PY{l+s+s2}{\PYZdq{}}\PY{p}{)}
		\PY{n+nb}{print}\PY{p}{(}\PY{n}{matrix2}\PY{p}{)}

		\PY{n}{result} \PY{o}{=} \PY{n}{np}\PY{o}{.}\PY{n}{dot}\PY{p}{(}\PY{n}{matrix1}\PY{p}{,} \PY{n}{matrix2}\PY{p}{)}

		\PY{n+nb}{print}\PY{p}{(}\PY{l+s+s2}{\PYZdq{}}\PY{l+s+s2}{Результат множення:}\PY{l+s+s2}{\PYZdq{}}\PY{p}{)}
		\PY{n+nb}{print}\PY{p}{(}\PY{n}{result}\PY{p}{)}
	\end{Verbatim}
\end{tcolorbox}

\begin{Verbatim}[commandchars=\\\{\}]
	Матриця 1:
	[[ 8  6]
		[10  4]]
	Матриця 2:
	[[ 7 10  4]
		[ 6 10  7]]
	Результат множення:
	[[ 92 140  74]
		[ 94 140  68]]
\end{Verbatim}

\section{\texorpdfstring{Створіть матрицю розміром \(3\times3\),
		заповніть її випадковими цілими числами в діапазоні від 1 до 10 та
		знайдіть її обернену
		матрицю.}{Створіть матрицю розміром 3\textbackslash times3, заповніть її випадковими цілими числами в діапазоні від 1 до 10 та знайдіть її обернену матрицю.}}

Обсичлення оберненої матриці здійснюється за формулою:

\[ \mathrm{A}^{-1} = \frac{1}{\mathrm{det}(\mathrm{A})} \cdot \mathrm{adj}(\mathrm{A}), \]

де $ \mathrm{det}(A) $ --- визначник матриці \(\mathrm A\),
\(\mathrm{adj}(\mathrm{A})\) --- доповнена матриця.

Обернена матриця існує, якщо $ \mathrm{det}(A) \neq 0$. Матриця, для
якої $ \mathrm{det}(A) = 0$ називається сингулярною, і для такої
матриці оберненої не існує. Тому в коді ми маємо це перевірити.

\begin{tcolorbox}[breakable, size=fbox, boxrule=1pt, pad at break*=1mm,colback=cellbackground, colframe=cellborder]
	\prompt{In}{incolor}{11}{\boxspacing}
	\begin{Verbatim}[commandchars=\\\{\}]
		\PY{c+c1}{\PYZsh{} Створимо матрицю розміром 3x3 з випадковими цілими числами від 1 до 10}
		\PY{n}{matrix} \PY{o}{=} \PY{n}{np}\PY{o}{.}\PY{n}{random}\PY{o}{.}\PY{n}{randint}\PY{p}{(}\PY{l+m+mi}{1}\PY{p}{,} \PY{l+m+mi}{11}\PY{p}{,} \PY{p}{(}\PY{l+m+mi}{3}\PY{p}{,} \PY{l+m+mi}{3}\PY{p}{)}\PY{p}{)}

		\PY{c+c1}{\PYZsh{} Виведемо початкову матрицю}
		\PY{n+nb}{print}\PY{p}{(}\PY{l+s+s2}{\PYZdq{}}\PY{l+s+s2}{Початкова матриця:}\PY{l+s+s2}{\PYZdq{}}\PY{p}{)}
		\PY{n+nb}{print}\PY{p}{(}\PY{n}{matrix}\PY{p}{)}

		\PY{c+c1}{\PYZsh{} Знайдемо обернену матрицю}
		\PY{k}{try}\PY{p}{:}
		\PY{n}{inverse\PYZus{}matrix} \PY{o}{=} \PY{n}{np}\PY{o}{.}\PY{n}{linalg}\PY{o}{.}\PY{n}{inv}\PY{p}{(}\PY{n}{matrix}\PY{p}{)}
		\PY{n+nb}{print}\PY{p}{(}\PY{l+s+s2}{\PYZdq{}}\PY{l+s+s2}{Обернена матриця:}\PY{l+s+s2}{\PYZdq{}}\PY{p}{)}
		\PY{n+nb}{print}\PY{p}{(}\PY{n}{inverse\PYZus{}matrix}\PY{p}{)}
		\PY{k}{except} \PY{n}{np}\PY{o}{.}\PY{n}{linalg}\PY{o}{.}\PY{n}{LinAlgError}\PY{p}{:}
		\PY{n+nb}{print}\PY{p}{(}\PY{l+s+s2}{\PYZdq{}}\PY{l+s+s2}{Матриця не має оберненої матриці, оскільки вона може бути сингулярною.}\PY{l+s+s2}{\PYZdq{}}\PY{p}{)}
	\end{Verbatim}
\end{tcolorbox}

\begin{Verbatim}[commandchars=\\\{\}]
	Початкова матриця:
	[[ 2  4  6]
		[ 7  1  1]
		[ 5  1 10]]
	Обернена матриця:
	[[-0.03913043  0.14782609  0.00869565]
		[ 0.2826087   0.04347826 -0.17391304]
		[-0.00869565 -0.07826087  0.11304348]]
\end{Verbatim}

\section{Створіть матрицю розміром 4x4, заповніть її випадковими дійсними числами в діапазоні від 0 до 1 та транспонуйте її.}

Транспонована матриця \(\mathrm{A}^T\), виникає з матриці \(\mathrm{A}\)
врезультаті заміни рядків на стлвбчики $(\mathrm{A}^{T})_{ji} =
	\mathrm{A}_{ij} $:

\[A =
	\begin{bmatrix}
		1 & 2 & 3 \\
		4 & 5 & 6 \\
	\end{bmatrix}
\]

Тоді транспонована матриця $ A^{T} $ буде мати вигляд:

\[
	A^T =
	\begin{bmatrix}
		1 & 4 \\
		2 & 5 \\
		3 & 6 \\
	\end{bmatrix}
\]

\begin{tcolorbox}[breakable, size=fbox, boxrule=1pt, pad at break*=1mm,colback=cellbackground, colframe=cellborder]
	\prompt{In}{incolor}{12}{\boxspacing}
	\begin{Verbatim}[commandchars=\\\{\}]
		\PY{n}{matrix} \PY{o}{=} \PY{n}{np}\PY{o}{.}\PY{n}{random}\PY{o}{.}\PY{n}{rand}\PY{p}{(}\PY{l+m+mi}{4}\PY{p}{,} \PY{l+m+mi}{4}\PY{p}{)}

		\PY{c+c1}{\PYZsh{} Виведемо початкову матрицю}
		\PY{n+nb}{print}\PY{p}{(}\PY{l+s+s2}{\PYZdq{}}\PY{l+s+s2}{Початкова матриця:}\PY{l+s+s2}{\PYZdq{}}\PY{p}{)}
		\PY{n+nb}{print}\PY{p}{(}\PY{n}{matrix}\PY{p}{)}

		\PY{c+c1}{\PYZsh{} Транспонуємо матрицю}
		\PY{n}{transposed\PYZus{}matrix} \PY{o}{=} \PY{n}{np}\PY{o}{.}\PY{n}{transpose}\PY{p}{(}\PY{n}{matrix}\PY{p}{)}

		\PY{c+c1}{\PYZsh{} Виведемо транспоновану матрицю}
		\PY{n+nb}{print}\PY{p}{(}\PY{l+s+s2}{\PYZdq{}}\PY{l+s+s2}{Транспонована матриця:}\PY{l+s+s2}{\PYZdq{}}\PY{p}{)}
		\PY{n+nb}{print}\PY{p}{(}\PY{n}{transposed\PYZus{}matrix}\PY{p}{)}
	\end{Verbatim}
\end{tcolorbox}

\begin{Verbatim}[commandchars=\\\{\}]
	Початкова матриця:
	[[0.08402274 0.07052606 0.94668645 0.47278038]
		[0.65529227 0.45497744 0.38751573 0.46614788]
		[0.57033985 0.4134212  0.46341259 0.69720946]
		[0.15384283 0.5854396  0.90123728 0.15706181]]
	Транспонована матриця:
	[[0.08402274 0.65529227 0.57033985 0.15384283]
		[0.07052606 0.45497744 0.4134212  0.5854396 ]
		[0.94668645 0.38751573 0.46341259 0.90123728]
		[0.47278038 0.46614788 0.69720946 0.15706181]]
\end{Verbatim}

\section{\texorpdfstring{Створіть матрицю розміром \(3\times4\) та вектор розміром 4, заповніть їх випадковими цілими числами в діапазоні від 1 до 10 та перемножте матрицю на
		вектор.}{Створіть матрицю розміром 3\textbackslash times4 та вектор розміром 4, заповніть їх випадковими цілими числами в діапазоні від 1 до 10 та перемножте матрицю на вектор.}}
\[
	\mathbf{u} = \mathrm{A} \cdot \mathbf{v} =
	\begin{bmatrix}
		a_{11} & a_{12} & a_{13} \\
		a_{21} & a_{22} & a_{23} \\
		a_{31} & a_{32} & a_{33} \\
	\end{bmatrix}
	\begin{bmatrix}
		v_1 \\
		v_2 \\
		v_3 \\
	\end{bmatrix}
	=
	\begin{bmatrix}
		a_{11} \cdot v_1 + a_{12} \cdot v_2 + a_{13} \cdot v_3 \\
		a_{21} \cdot v_1 + a_{22} \cdot v_2 + a_{23} \cdot v_3 \\
		a_{31} \cdot v_1 + a_{32} \cdot v_2 + a_{33} \cdot v_3 \\
	\end{bmatrix}
\]

\begin{tcolorbox}[breakable, size=fbox, boxrule=1pt, pad at break*=1mm,colback=cellbackground, colframe=cellborder]
	\prompt{In}{incolor}{13}{\boxspacing}
	\begin{Verbatim}[commandchars=\\\{\}]
		\PY{c+c1}{\PYZsh{} Створимо матрицю розміром 3x4 з випадковими цілими числами від 1 до 10}
		\PY{n}{matrix} \PY{o}{=} \PY{n}{np}\PY{o}{.}\PY{n}{random}\PY{o}{.}\PY{n}{randint}\PY{p}{(}\PY{l+m+mi}{1}\PY{p}{,} \PY{l+m+mi}{11}\PY{p}{,} \PY{p}{(}\PY{l+m+mi}{3}\PY{p}{,} \PY{l+m+mi}{4}\PY{p}{)}\PY{p}{)}

		\PY{c+c1}{\PYZsh{} Створимо вектор розміром 4 з випадковими цілими числами від 1 до 10}
		\PY{n}{vector} \PY{o}{=} \PY{n}{np}\PY{o}{.}\PY{n}{random}\PY{o}{.}\PY{n}{randint}\PY{p}{(}\PY{l+m+mi}{1}\PY{p}{,} \PY{l+m+mi}{11}\PY{p}{,} \PY{l+m+mi}{4}\PY{p}{)}

		\PY{c+c1}{\PYZsh{} Виведемо матрицю і вектор}
		\PY{n+nb}{print}\PY{p}{(}\PY{l+s+s2}{\PYZdq{}}\PY{l+s+s2}{Матриця:}\PY{l+s+s2}{\PYZdq{}}\PY{p}{)}
		\PY{n+nb}{print}\PY{p}{(}\PY{n}{matrix}\PY{p}{)}
		\PY{n+nb}{print}\PY{p}{(}\PY{l+s+s2}{\PYZdq{}}\PY{l+s+s2}{Вектор:}\PY{l+s+s2}{\PYZdq{}}\PY{p}{)}
		\PY{n+nb}{print}\PY{p}{(}\PY{n}{vector}\PY{p}{)}

		\PY{c+c1}{\PYZsh{} Перемножимо матрицю на вектор}
		\PY{n}{result} \PY{o}{=} \PY{n}{np}\PY{o}{.}\PY{n}{dot}\PY{p}{(}\PY{n}{matrix}\PY{p}{,} \PY{n}{vector}\PY{p}{)}

		\PY{c+c1}{\PYZsh{} Виведемо результат}
		\PY{n+nb}{print}\PY{p}{(}\PY{l+s+s2}{\PYZdq{}}\PY{l+s+s2}{Результат перемноження:}\PY{l+s+s2}{\PYZdq{}}\PY{p}{)}
		\PY{n+nb}{print}\PY{p}{(}\PY{n}{result}\PY{p}{)}
	\end{Verbatim}
\end{tcolorbox}

\begin{Verbatim}[commandchars=\\\{\}]
	Матриця:
	[[5 3 6 2]
		[7 2 7 7]
		[5 9 5 7]]
	Вектор:
	[ 8  3 10  9]
	Результат перемноження:
	[127 195 180]
\end{Verbatim}

\section{\texorpdfstring{Створіть матрицю розміром \(2\times3\)
		та вектор розміром 3, заповніть їх випадковими дійсними числами в
		діапазоні від 0 до 1 та перемножте матрицю на
		вектор.}{Створіть матрицю розміром 2\textbackslash times3 та вектор розміром 3, заповніть їх випадковими дійсними числами в діапазоні від 0 до 1 та перемножте матрицю на вектор.}}\label{ux441ux442ux432ux43eux440ux456ux442ux44c-ux43cux430ux442ux440ux438ux446ux44e-ux440ux43eux437ux43cux456ux440ux43eux43c-2times3-ux442ux430-ux432ux435ux43aux442ux43eux440-ux440ux43eux437ux43cux456ux440ux43eux43c-3-ux437ux430ux43fux43eux432ux43dux456ux442ux44c-ux457ux445-ux432ux438ux43fux430ux434ux43aux43eux432ux438ux43cux438-ux434ux456ux439ux441ux43dux438ux43cux438-ux447ux438ux441ux43bux430ux43cux438-ux432-ux434ux456ux430ux43fux430ux437ux43eux43dux456-ux432ux456ux434-0-ux434ux43e-1-ux442ux430-ux43fux435ux440ux435ux43cux43dux43eux436ux442ux435-ux43cux430ux442ux440ux438ux446ux44e-ux43dux430-ux432ux435ux43aux442ux43eux440.}

\begin{tcolorbox}[breakable, size=fbox, boxrule=1pt, pad at break*=1mm,colback=cellbackground, colframe=cellborder]
	\prompt{In}{incolor}{14}{\boxspacing}
	\begin{Verbatim}[commandchars=\\\{\}]
		\PY{c+c1}{\PYZsh{} Створимо матрицю розміром 2x3 з випадковими дійсними числами від 0 до 1}
		\PY{n}{matrix} \PY{o}{=} \PY{n}{np}\PY{o}{.}\PY{n}{random}\PY{o}{.}\PY{n}{rand}\PY{p}{(}\PY{l+m+mi}{2}\PY{p}{,} \PY{l+m+mi}{3}\PY{p}{)}

		\PY{c+c1}{\PYZsh{} Створимо вектор розміром 3 з випадковими дійсними числами від 0 до 1}
		\PY{n}{vector} \PY{o}{=} \PY{n}{np}\PY{o}{.}\PY{n}{random}\PY{o}{.}\PY{n}{rand}\PY{p}{(}\PY{l+m+mi}{3}\PY{p}{)}

		\PY{c+c1}{\PYZsh{} Виведемо матрицю і вектор}
		\PY{n+nb}{print}\PY{p}{(}\PY{l+s+s2}{\PYZdq{}}\PY{l+s+s2}{Матриця:}\PY{l+s+s2}{\PYZdq{}}\PY{p}{)}
		\PY{n+nb}{print}\PY{p}{(}\PY{n}{matrix}\PY{p}{)}
		\PY{n+nb}{print}\PY{p}{(}\PY{l+s+s2}{\PYZdq{}}\PY{l+s+s2}{Вектор:}\PY{l+s+s2}{\PYZdq{}}\PY{p}{)}
		\PY{n+nb}{print}\PY{p}{(}\PY{n}{vector}\PY{p}{)}

		\PY{c+c1}{\PYZsh{} Перемножимо матрицю на вектор}
		\PY{n}{result} \PY{o}{=} \PY{n}{np}\PY{o}{.}\PY{n}{dot}\PY{p}{(}\PY{n}{matrix}\PY{p}{,} \PY{n}{vector}\PY{p}{)}

		\PY{c+c1}{\PYZsh{} Виведемо результат}
		\PY{n+nb}{print}\PY{p}{(}\PY{l+s+s2}{\PYZdq{}}\PY{l+s+s2}{Результат перемноження:}\PY{l+s+s2}{\PYZdq{}}\PY{p}{)}
		\PY{n+nb}{print}\PY{p}{(}\PY{n}{result}\PY{p}{)}
	\end{Verbatim}
\end{tcolorbox}

\begin{Verbatim}[commandchars=\\\{\}]
	Матриця:
	[[0.40142629 0.24681107 0.41057536]
		[0.72516117 0.91335118 0.68146542]]
	Вектор:
	[0.3660769  0.20024095 0.43328539]
	Результат перемноження:
	[0.37427088 0.74362408]
\end{Verbatim}

\section{Створіть дві матриці розміром 2x2, заповніть їх
	випадковими цілими числами в діапазоні від 1 до 10 та виконайте їхнє
	поелементне
	множення.}

Поелементне множення двох матриць визначається так: результатом цієї
операції є нова матриця, у якій кожний елемент результуючої матриці
визначається, як добуток відповідних елементів вихідних матриць.

Елемент результуючої матриці \(\mathrm C\) обчислюється як добуток
відповідних елементів матриць \(\mathrm A\) і \(\mathrm B\):

\[
	\mathrm C_{ij}  = \mathrm A_{ij} \cdot \mathrm B_{ij}
\]

де \(\mathrm C_{ij}\) --- елемент результуючої матриці \(\mathrm C\) на
позиції \((i, j)\), \(\mathrm A_{ij}\) --- елемент матриці \(\mathrm A\)
на позиції \((i, j)\) і \(\mathrm B_{ij}\) - елемент матриці
\(\mathrm B\) на позиції \((i, j)\).

Поелементне множення відрізняється від стандартного множення матриць,
оскільки в останньому випадку обчислюється добуток рядків однієї матриці
на стовпці іншої матриці, що веде до іншої результуючої матриці.

\begin{tcolorbox}[breakable, size=fbox, boxrule=1pt, pad at break*=1mm,colback=cellbackground, colframe=cellborder]
	\prompt{In}{incolor}{15}{\boxspacing}
	\begin{Verbatim}[commandchars=\\\{\}]
		\PY{c+c1}{\PYZsh{} Створимо першу матрицю 2x2 з випадковими цілими числами в діапазоні від 1 до 10}
		\PY{n}{matrix1} \PY{o}{=} \PY{n}{np}\PY{o}{.}\PY{n}{random}\PY{o}{.}\PY{n}{randint}\PY{p}{(}\PY{l+m+mi}{1}\PY{p}{,} \PY{l+m+mi}{11}\PY{p}{,} \PY{p}{(}\PY{l+m+mi}{2}\PY{p}{,} \PY{l+m+mi}{2}\PY{p}{)}\PY{p}{)}

		\PY{c+c1}{\PYZsh{} Створимо другу матрицю 2x2 з випадковими цілими числами в діапазоні від 1 до 10}
		\PY{n}{matrix2} \PY{o}{=} \PY{n}{np}\PY{o}{.}\PY{n}{random}\PY{o}{.}\PY{n}{randint}\PY{p}{(}\PY{l+m+mi}{1}\PY{p}{,} \PY{l+m+mi}{11}\PY{p}{,} \PY{p}{(}\PY{l+m+mi}{2}\PY{p}{,} \PY{l+m+mi}{2}\PY{p}{)}\PY{p}{)}

		\PY{c+c1}{\PYZsh{} Виведемо обидві матриці}
		\PY{n+nb}{print}\PY{p}{(}\PY{l+s+s2}{\PYZdq{}}\PY{l+s+s2}{Перша матриця:}\PY{l+s+s2}{\PYZdq{}}\PY{p}{)}
		\PY{n+nb}{print}\PY{p}{(}\PY{n}{matrix1}\PY{p}{)}
		\PY{n+nb}{print}\PY{p}{(}\PY{l+s+s2}{\PYZdq{}}\PY{l+s+s2}{Друга матриця:}\PY{l+s+s2}{\PYZdq{}}\PY{p}{)}
		\PY{n+nb}{print}\PY{p}{(}\PY{n}{matrix2}\PY{p}{)}

		\PY{c+c1}{\PYZsh{} Виконаємо поелементне множення матриць}
		\PY{n}{result} \PY{o}{=} \PY{n}{np}\PY{o}{.}\PY{n}{multiply}\PY{p}{(}\PY{n}{matrix1}\PY{p}{,} \PY{n}{matrix2}\PY{p}{)}

		\PY{c+c1}{\PYZsh{} Виведемо результат}
		\PY{n+nb}{print}\PY{p}{(}\PY{l+s+s2}{\PYZdq{}}\PY{l+s+s2}{Результат поелементного множення:}\PY{l+s+s2}{\PYZdq{}}\PY{p}{)}
		\PY{n+nb}{print}\PY{p}{(}\PY{n}{result}\PY{p}{)}
	\end{Verbatim}
\end{tcolorbox}

\begin{Verbatim}[commandchars=\\\{\}]
	Перша матриця:
	[[5 7]
		[1 2]]
	Друга матриця:
	[[ 7 10]
		[ 3  9]]
	Результат поелементного множення:
	[[35 70]
		[ 3 18]]
\end{Verbatim}

\section{\texorpdfstring{Створіть дві матриці розміром
		\(2\times 2\), заповніть їх випадковими цілими числами в діапазоні від 1
		до 10 та знайдіть їх
		добуток.}{Створіть дві матриці розміром 2\textbackslash times 2, заповніть їх випадковими цілими числами в діапазоні від 1 до 10 та знайдіть їх добуток.}}

Добуток матриць визначаеться як:

\[\mathrm C_{ij} = \sum_{r=1}^m \mathrm A_{ir} \mathrm B_{rj} \;\;\; \left(i=1, 2, \ldots l;\; j=1, 2, \ldots n \right)\]

\begin{tcolorbox}[breakable, size=fbox, boxrule=1pt, pad at break*=1mm,colback=cellbackground, colframe=cellborder]
	\prompt{In}{incolor}{16}{\boxspacing}
	\begin{Verbatim}[commandchars=\\\{\}]
		\PY{c+c1}{\PYZsh{} Створимо першу матрицю 2x2 з випадковими цілими числами в діапазоні від 1 до 10}
		\PY{n}{matrix1} \PY{o}{=} \PY{n}{np}\PY{o}{.}\PY{n}{random}\PY{o}{.}\PY{n}{randint}\PY{p}{(}\PY{l+m+mi}{1}\PY{p}{,} \PY{l+m+mi}{11}\PY{p}{,} \PY{p}{(}\PY{l+m+mi}{2}\PY{p}{,} \PY{l+m+mi}{2}\PY{p}{)}\PY{p}{)}

		\PY{c+c1}{\PYZsh{} Створимо другу матрицю 2x2 з випадковими цілими числами в діапазоні від 1 до 10}
		\PY{n}{matrix2} \PY{o}{=} \PY{n}{np}\PY{o}{.}\PY{n}{random}\PY{o}{.}\PY{n}{randint}\PY{p}{(}\PY{l+m+mi}{1}\PY{p}{,} \PY{l+m+mi}{11}\PY{p}{,} \PY{p}{(}\PY{l+m+mi}{2}\PY{p}{,} \PY{l+m+mi}{2}\PY{p}{)}\PY{p}{)}

		\PY{c+c1}{\PYZsh{} Виведемо обидві матриці}
		\PY{n+nb}{print}\PY{p}{(}\PY{l+s+s2}{\PYZdq{}}\PY{l+s+s2}{Перша матриця:}\PY{l+s+s2}{\PYZdq{}}\PY{p}{)}
		\PY{n+nb}{print}\PY{p}{(}\PY{n}{matrix1}\PY{p}{)}
		\PY{n+nb}{print}\PY{p}{(}\PY{l+s+s2}{\PYZdq{}}\PY{l+s+s2}{Друга матриця:}\PY{l+s+s2}{\PYZdq{}}\PY{p}{)}
		\PY{n+nb}{print}\PY{p}{(}\PY{n}{matrix2}\PY{p}{)}

		\PY{c+c1}{\PYZsh{} Знайдемо добуток матриць}
		\PY{n}{result} \PY{o}{=} \PY{n}{np}\PY{o}{.}\PY{n}{dot}\PY{p}{(}\PY{n}{matrix1}\PY{p}{,} \PY{n}{matrix2}\PY{p}{)}

		\PY{c+c1}{\PYZsh{} Виведемо результат}
		\PY{n+nb}{print}\PY{p}{(}\PY{l+s+s2}{\PYZdq{}}\PY{l+s+s2}{Результат добутку матриць:}\PY{l+s+s2}{\PYZdq{}}\PY{p}{)}
		\PY{n+nb}{print}\PY{p}{(}\PY{n}{result}\PY{p}{)}
	\end{Verbatim}
\end{tcolorbox}

\begin{Verbatim}[commandchars=\\\{\}]
	Перша матриця:
	[[1 3]
		[2 5]]
	Друга матриця:
	[[7 7]
		[2 3]]
	Результат добутку матриць:
	[[13 16]
		[24 29]]
\end{Verbatim}

\section{\texorpdfstring{Створіть матрицю розміром \(5\times 5\),
		заповніть її випадковими цілими числами в діапазоні від 1 до 100 та
		знайдіть суму елементів
		матриці.}{Створіть матрицю розміром 5\textbackslash times 5, заповніть її випадковими цілими числами в діапазоні від 1 до 100 та знайдіть суму елементів матриці.}}

\begin{tcolorbox}[breakable, size=fbox, boxrule=1pt, pad at break*=1mm,colback=cellbackground, colframe=cellborder]
	\prompt{In}{incolor}{17}{\boxspacing}
	\begin{Verbatim}[commandchars=\\\{\}]
		\PY{c+c1}{\PYZsh{} Створимо матрицю 5x5 з випадковими цілими числами від 1 до 100}
		\PY{n}{matrix} \PY{o}{=} \PY{n}{np}\PY{o}{.}\PY{n}{random}\PY{o}{.}\PY{n}{randint}\PY{p}{(}\PY{l+m+mi}{1}\PY{p}{,} \PY{l+m+mi}{101}\PY{p}{,} \PY{p}{(}\PY{l+m+mi}{5}\PY{p}{,} \PY{l+m+mi}{5}\PY{p}{)}\PY{p}{)}

		\PY{c+c1}{\PYZsh{} Виведемо матрицю}
		\PY{n+nb}{print}\PY{p}{(}\PY{l+s+s2}{\PYZdq{}}\PY{l+s+s2}{Матриця:}\PY{l+s+s2}{\PYZdq{}}\PY{p}{)}
		\PY{n+nb}{print}\PY{p}{(}\PY{n}{matrix}\PY{p}{)}

		\PY{c+c1}{\PYZsh{} Знайдемо суму елементів матриці}
		\PY{n}{sum\PYZus{}of\PYZus{}elements} \PY{o}{=} \PY{n}{np}\PY{o}{.}\PY{n}{sum}\PY{p}{(}\PY{n}{matrix}\PY{p}{)}

		\PY{c+c1}{\PYZsh{} Виведемо суму}
		\PY{n+nb}{print}\PY{p}{(}\PY{l+s+s2}{\PYZdq{}}\PY{l+s+s2}{Сума елементів матриці: }\PY{l+s+s2}{\PYZdq{}}\PY{p}{,} \PY{n}{sum\PYZus{}of\PYZus{}elements}\PY{p}{)}
	\end{Verbatim}
\end{tcolorbox}

\begin{Verbatim}[commandchars=\\\{\}]
	Матриця:
	[[ 65  54  46 100  89]
		[ 19  41  79  28  70]
		[ 73  27  94  20  94]
		[ 46  35  18  68  27]
		[ 36  63  33  76   7]]
	Сума елементів матриці:  1308
\end{Verbatim}

\section{\texorpdfstring{Створіть дві матриці розміром
		\(4\times 4\), заповніть їх випадковими цілими числами в діапазоні від 1
		до 10 та знайдіть їхню
		різницю.}{Створіть дві матриці розміром 4\textbackslash times 4, заповніть їх випадковими цілими числами в діапазоні від 1 до 10 та знайдіть їхню різницю.}}

Віднімання матриць \(\mathrm A - \mathrm B\) - це операція обчислення
матриці \(\mathrm C\) , усі елементи якої дорівнюють попарній різниці
всіх відповідних елементів матриць \(\mathrm A\) і \(\mathrm B\), тобто
фактичноце полементне віднімання.

\begin{tcolorbox}[breakable, size=fbox, boxrule=1pt, pad at break*=1mm,colback=cellbackground, colframe=cellborder]
	\prompt{In}{incolor}{18}{\boxspacing}
	\begin{Verbatim}[commandchars=\\\{\}]
		\PY{c+c1}{\PYZsh{} Створимо першу матрицю 4x4 з випадковими цілими числами від 1 до 10}
		\PY{n}{matrix1} \PY{o}{=} \PY{n}{np}\PY{o}{.}\PY{n}{random}\PY{o}{.}\PY{n}{randint}\PY{p}{(}\PY{l+m+mi}{1}\PY{p}{,} \PY{l+m+mi}{11}\PY{p}{,} \PY{p}{(}\PY{l+m+mi}{4}\PY{p}{,} \PY{l+m+mi}{4}\PY{p}{)}\PY{p}{)}

		\PY{c+c1}{\PYZsh{} Створимо другу матрицю 4x4 з випадковими цілими числами від 1 до 10}
		\PY{n}{matrix2} \PY{o}{=} \PY{n}{np}\PY{o}{.}\PY{n}{random}\PY{o}{.}\PY{n}{randint}\PY{p}{(}\PY{l+m+mi}{1}\PY{p}{,} \PY{l+m+mi}{11}\PY{p}{,} \PY{p}{(}\PY{l+m+mi}{4}\PY{p}{,} \PY{l+m+mi}{4}\PY{p}{)}\PY{p}{)}

		\PY{c+c1}{\PYZsh{} Виведемо обидві матриці}
		\PY{n+nb}{print}\PY{p}{(}\PY{l+s+s2}{\PYZdq{}}\PY{l+s+s2}{Перша матриця:}\PY{l+s+s2}{\PYZdq{}}\PY{p}{)}
		\PY{n+nb}{print}\PY{p}{(}\PY{n}{matrix1}\PY{p}{)}
		\PY{n+nb}{print}\PY{p}{(}\PY{l+s+s2}{\PYZdq{}}\PY{l+s+s2}{Друга матриця:}\PY{l+s+s2}{\PYZdq{}}\PY{p}{)}
		\PY{n+nb}{print}\PY{p}{(}\PY{n}{matrix2}\PY{p}{)}

		\PY{c+c1}{\PYZsh{} Знайдемо різницю матриць}
		\PY{n}{result} \PY{o}{=} \PY{n}{matrix1} \PY{o}{\PYZhy{}} \PY{n}{matrix2}

		\PY{c+c1}{\PYZsh{} Виведемо результат}
		\PY{n+nb}{print}\PY{p}{(}\PY{l+s+s2}{\PYZdq{}}\PY{l+s+s2}{Різниця матриць:}\PY{l+s+s2}{\PYZdq{}}\PY{p}{)}
		\PY{n+nb}{print}\PY{p}{(}\PY{n}{result}\PY{p}{)}
	\end{Verbatim}
\end{tcolorbox}

\begin{Verbatim}[commandchars=\\\{\}]
	Перша матриця:
	[[ 6  4 10  4]
		[ 6  3  7 10]
		[ 5  9  6  4]
		[ 5  9 10  5]]
	Друга матриця:
	[[6 6 4 2]
		[6 1 2 8]
		[2 9 1 5]
		[5 3 5 5]]
	Різниця матриць:
	[[ 0 -2  6  2]
		[ 0  2  5  2]
		[ 3  0  5 -1]
		[ 0  6  5  0]]
\end{Verbatim}

\section{\texorpdfstring{Створіть матрицю розміром \(3\times 3\),
		заповніть її випадковими дійсними числами в діапазоні від 0 до 1 та
		знайдіть вектор-стовпчик, що містить суму елементів кожного рядка
		матриці.}{Створіть матрицю розміром 3\textbackslash times 3, заповніть її випадковими дійсними числами в діапазоні від 0 до 1 та знайдіть вектор-стовпчик, що містить суму елементів кожного рядка матриці.}}

\begin{tcolorbox}[breakable, size=fbox, boxrule=1pt, pad at break*=1mm,colback=cellbackground, colframe=cellborder]
	\prompt{In}{incolor}{19}{\boxspacing}
	\begin{Verbatim}[commandchars=\\\{\}]
		\PY{c+c1}{\PYZsh{} Створимо матрицю 3x3 з випадковими дійсними числами в діапазоні від 0 до 1}
		\PY{n}{matrix} \PY{o}{=} \PY{n}{np}\PY{o}{.}\PY{n}{random}\PY{o}{.}\PY{n}{rand}\PY{p}{(}\PY{l+m+mi}{3}\PY{p}{,} \PY{l+m+mi}{3}\PY{p}{)}

		\PY{c+c1}{\PYZsh{} Виведемо матрицю}
		\PY{n+nb}{print}\PY{p}{(}\PY{l+s+s2}{\PYZdq{}}\PY{l+s+s2}{Матриця:}\PY{l+s+s2}{\PYZdq{}}\PY{p}{)}
		\PY{n+nb}{print}\PY{p}{(}\PY{n}{matrix}\PY{p}{)}

		\PY{c+c1}{\PYZsh{} Знайдемо суму елементів кожного рядка матриці та створимо вектор\PYZhy{}стовпчик}
		\PY{n}{sum\PYZus{}of\PYZus{}rows} \PY{o}{=} \PY{n}{np}\PY{o}{.}\PY{n}{sum}\PY{p}{(}\PY{n}{matrix}\PY{p}{,} \PY{n}{axis}\PY{o}{=}\PY{l+m+mi}{1}\PY{p}{,} \PY{n}{keepdims}\PY{o}{=}\PY{k+kc}{True}\PY{p}{)}

		\PY{c+c1}{\PYZsh{} Виведемо вектор\PYZhy{}стовпчик із сумами}
		\PY{n+nb}{print}\PY{p}{(}\PY{l+s+s2}{\PYZdq{}}\PY{l+s+s2}{Вектор\PYZhy{}стовпчик із сумами рядків:}\PY{l+s+s2}{\PYZdq{}}\PY{p}{)}
		\PY{n+nb}{print}\PY{p}{(}\PY{n}{sum\PYZus{}of\PYZus{}rows}\PY{p}{)}
	\end{Verbatim}
\end{tcolorbox}

\begin{Verbatim}[commandchars=\\\{\}]
	Матриця:
	[[0.01297979 0.47161382 0.59052176]
		[0.55785916 0.35450255 0.41946586]
		[0.67527101 0.41623059 0.6897199 ]]
	Вектор-стовпчик із сумами рядків:
	[[1.07511538]
		[1.33182756]
		[1.78122151]]
\end{Verbatim}

\section{\texorpdfstring{Створіть матрицю розміром \(3\times4\) з
		довільними цілими числами і створінь матрицю з квадратами цих
		чисел.}{Створіть матрицю розміром 3\textbackslash times4 з довільними цілими числами і створінь матрицю з квадратами цих чисел.}}

\begin{tcolorbox}[breakable, size=fbox, boxrule=1pt, pad at break*=1mm,colback=cellbackground, colframe=cellborder]
	\prompt{In}{incolor}{20}{\boxspacing}
	\begin{Verbatim}[commandchars=\\\{\}]
		\PY{c+c1}{\PYZsh{} Створимо матрицю 3x4 з довільними цілими числами (від \PYZhy{}10 до 10)}
		\PY{n}{matrix} \PY{o}{=} \PY{n}{np}\PY{o}{.}\PY{n}{random}\PY{o}{.}\PY{n}{randint}\PY{p}{(}\PY{o}{\PYZhy{}}\PY{l+m+mi}{10}\PY{p}{,} \PY{l+m+mi}{11}\PY{p}{,} \PY{p}{(}\PY{l+m+mi}{3}\PY{p}{,} \PY{l+m+mi}{4}\PY{p}{)}\PY{p}{)}

		\PY{c+c1}{\PYZsh{} Виведемо початкову матрицю}
		\PY{n+nb}{print}\PY{p}{(}\PY{l+s+s2}{\PYZdq{}}\PY{l+s+s2}{Початкова матриця:}\PY{l+s+s2}{\PYZdq{}}\PY{p}{)}
		\PY{n+nb}{print}\PY{p}{(}\PY{n}{matrix}\PY{p}{)}

		\PY{c+c1}{\PYZsh{} Знайдемо матрицю з квадратами чисел}
		\PY{n}{squared\PYZus{}matrix} \PY{o}{=} \PY{n}{np}\PY{o}{.}\PY{n}{square}\PY{p}{(}\PY{n}{matrix}\PY{p}{)}

		\PY{c+c1}{\PYZsh{} Виведемо матрицю з квадратами чисел}
		\PY{n+nb}{print}\PY{p}{(}\PY{l+s+s2}{\PYZdq{}}\PY{l+s+s2}{Матриця з квадратами чисел:}\PY{l+s+s2}{\PYZdq{}}\PY{p}{)}
		\PY{n+nb}{print}\PY{p}{(}\PY{n}{squared\PYZus{}matrix}\PY{p}{)}
	\end{Verbatim}
\end{tcolorbox}

\begin{Verbatim}[commandchars=\\\{\}]
	Початкова матриця:
	[[-7 -9 -5  3]
		[-2 -8 -6 10]
		[-1  2 -6  9]]
	Матриця з квадратами чисел:
	[[ 49  81  25   9]
		[  4  64  36 100]
		[  1   4  36  81]]
\end{Verbatim}

\section{Створіть вектор розміром 4, заповніть його випадковими
	цілими числами в діапазоні від 1 до 50 та знайдіть вектор з квадратними
	коренями цих
	чисел.}

\begin{tcolorbox}[breakable, size=fbox, boxrule=1pt, pad at break*=1mm,colback=cellbackground, colframe=cellborder]
	\prompt{In}{incolor}{21}{\boxspacing}
	\begin{Verbatim}[commandchars=\\\{\}]
		\PY{c+c1}{\PYZsh{} Створимо вектор розміром 4 з випадковими цілими числами від 1 до 50}
		\PY{n}{vector} \PY{o}{=} \PY{n}{np}\PY{o}{.}\PY{n}{random}\PY{o}{.}\PY{n}{randint}\PY{p}{(}\PY{l+m+mi}{1}\PY{p}{,} \PY{l+m+mi}{51}\PY{p}{,} \PY{l+m+mi}{4}\PY{p}{)}

		\PY{c+c1}{\PYZsh{} Виведемо початковий вектор}
		\PY{n+nb}{print}\PY{p}{(}\PY{l+s+s2}{\PYZdq{}}\PY{l+s+s2}{Початковий вектор:}\PY{l+s+s2}{\PYZdq{}}\PY{p}{,} \PY{n}{vector}\PY{p}{)}

		\PY{c+c1}{\PYZsh{} Знайдемо вектор з квадратними коренями чисел}
		\PY{n}{sqrt\PYZus{}vector} \PY{o}{=} \PY{n}{np}\PY{o}{.}\PY{n}{sqrt}\PY{p}{(}\PY{n}{vector}\PY{p}{)}

		\PY{c+c1}{\PYZsh{} Виведемо вектор з квадратними коренями}
		\PY{n+nb}{print}\PY{p}{(}\PY{l+s+s2}{\PYZdq{}}\PY{l+s+s2}{Вектор з квадратними коренями чисел:}\PY{l+s+s2}{\PYZdq{}}\PY{p}{,} \PY{n}{sqrt\PYZus{}vector}\PY{p}{)}
	\end{Verbatim}
\end{tcolorbox}

\begin{Verbatim}[commandchars=\\\{\}]
	Початковий вектор: [49 12 50 11]
	Вектор з квадратними коренями чисел: [7.         3.46410162 7.07106781
	3.31662479]
\end{Verbatim}


% Add a bibliography block to the postdoc



\end{document}
